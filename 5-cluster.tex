\chapter{Моделирование кластерной наноплазмы}
Данный раздел посвящен применению метода классической молекулярной динамики для моделирования кластерной наноплазмы.  Под кластером подразумевается наночастица произвольной формы с количеством атомов от $3$ до $10^7$. Свойства подобных частиц, особенно при малом количестве частиц, отличаются от свойств больших однородных жидкостей и твердых тел.
При изучении кластерной наноплазмы, особый интерес представляют динамические процессы в электронной подсистеме. Коллективные возбуждения электронов в такой плазме интересны в силу того, что они сильно взаимодействуют с внешними полями и влияют на распределение энергии в кластере. 

Одним из способов экспериментального изучения кластерной наноплазмы является использование фемтосекундных лазеров~\cite{Fennel-PRL07,Hickstein-PRL14}.
Для данной работы представляла особый интерес сильно неидеальная наноплазма, образованная путем воздействия лазерных импульсов интенсивностью $I=10^{12} - 10^{14}, \mathrm{\text{В}}/\mathrm{\text{см}}^2$~\cite{Reinholz_PRB06} на металлические нанокластеры.

Для теоретического изучения кластерной наноплазмы применим метод классической молекулярной динамики~\cite{NM-JETP05,RMRM-PRE04,MRRWZ-PRE05}.  В дополнение к термодинамическим свойствам плазмы, метод МД позволяет рассматривать транспортные свойства моделируемых систем, исходя из автокорреляционных функций и флуктуационно-диссипационной теоремы.  Метод МД не является единственным методом компьютерного моделирования, применимым для изучения кластерной наноплазмы. Так ранее рассмотренные методы молекулярной динамики с волновыми пакетами~\cite{GMMV-PRE13} и Монте-Карло в терминах интегралов по траекториям~\cite{filinov2004thermodynamic,Filinov-PRE15} обеспечивают более точный учет квантовых эффектов в наноплазме в отличии от классической МД, где они учитываются псевдопотенциалом. Но для моделирования кластерной наноплазмы, метод классической МД является наилучшим выбором. Благодаря малому количеству частиц в кластере обеспечивается большая длина траекторий что приводит к малым погрешностям вычисляемых величин. Высокие температуры (больше 1 эВ) обеспечивают сравнительно малый вклад квантовых эффектов в наблюдаемые величины и электронная динамика определяется в большинстве своем только кулоновскими силами. 

В работе рассматривалась электронная динамика в процессе ионизации кластера, включающая в себя: образования наноплазмы, вылет электронов из кластера, температурную релаксацию кластера, электронные колебания в равновесном состоянии. Размер кластеров варьировался от $55$ до $10^5$. Это позволило рассмотреть изменение свойств кластера с увеличением его размера.  В МД моделировании рассчитывался: минимальный заряд кластера в стационарном состоянии, изменение температуры кластера в процессе ионизации, спектр электронных колебаний в стационарном состоянии.

Приемлемой альтернативой методу МД  по производительности и моделируемым явлениям может быть метод MicPIC~\cite{Fennel-PRL12}. Этот метод включает в себя метод Particle in Cell на больших расстояниях, и метод классической молекулярной динамики. В связи с этим, появляется возможность моделировать кластеры больших размеров (более миллиона атомов)~\cite{Broda_CPP13,Gildenburg-PP2011}.

\section{Метод исследования релаксации кластерной наноплазмы}\label{sec:cluster-termalization}
Динамика электронной подсистемы кластерной наноплазмы включает следующие этапы: ионизация отдельных атомов и ионов вещества мишени, поглощение энергии ионизирующего
лазерного импульса, формирование и нагрев наноплазмы, эмиссия электронов из кластера,  эволюция температуры и рост заряда кластера со временем, установление квазиравновесного распределения плотности электронов внутри кластера, взаимодействие наноплазмы с повторным нагревающим импульсом или с диагностическим импульсом, следующим с задержкой в 0.1-1 пс. Ранее на основе МД моделирования получена универсальная зависимость заряда кластера от произведения его размера на температуру электронов; показано, что сдвиг частоты Ми в красную область спектра может быть объяснен неоднородностью радиального распределения электронной плотности.
Далее метод МД был применен для моделирования эволюции кластерной наноплазмы на характерных временах релаксации электронной подсистемы.

При моделировании кластерной нанопламы методом МД существовала проблема термализации кластера для получения начального распределения скоростей электронов в кластере. В используемых ранее подходах  термализация происходила в неограниченной системе, что допускало вылет электронов из кластера еще до достижения требуемой температуры. В связи с этим невозможно было определить, какое количество энергии передано системе на этапе термализации. В данной работе вылет электронов предотвращается заданием отражающего потенциала на небольшом удалении от поверхности кластера (рис. \ref{fig:cluster-Termalization}):
\begin{equation}
	U_{ext}(r)=-kr^2,~\text{для}~r = |\vec{r - r_0}| > R_i,
\end{equation}
где $\vec{r}$ --- положение электрона в пространстве, $\vec{r_0}$ --- центр кластера.
Параметр $k$ подбирается таким образом, чтобы обеспечить возврат частицы в ячейку за малое число шагов, но, в то же время, использование потенциала с этим параметром не должно приводить к нарушению сохранения полной энергии. Данный потенциал слабо влияет на термодинамические параметры кластера, и его исключение из расчета по достижению требуемой температуры не изменяет распределения электронных скоростей.

\begin{figure}[htb] %IM: Лучше заменить на [htb]
	\centering
	
	\includegraphics[width=1\linewidth]{figure/Cluster-Termalization.jpg}
	\caption{\label{fig:cluster-Termalization}Схематическое изображение алгоритма термализации.}
\end{figure}


\section{Сравнение модели точечных ионов и нейтрализующего фона}

Особенности моделируемой системы позволяют произвести несколько упрощений метода МД для уменьшения вычислительного времени. Первое допущение состоит в том, чтобы пренебречь движением ионной подсистемы, в связи с много большой массой ионов~\cite{RRRM-IJMPB08,Winkel-CPP13} (Restricted molecular dynamic (RMD)). В этом случае ионы располагаются в произвольном порядке внутри сферы. Произвольный порядок ионов позволяет устранить эффекты симметрии и провести усреднение наблюдаемых эффектов по различным ионным конфигурациям. Второе упрощение заключается в рассмотрении ионной подсистемы как однородной заряженной сферы~\cite{Broda_CPP13}. 
В этом случае все ионы заменяются эффективным потенциалом, определяемым радиусом кластера $R_i$ и концентрацией ионов $n_i$~\cite{Broda_CPP13}:
\begin{equation}\label{eq:cluster-potjellium}
U_\mathrm{Jellium}(r)=
\begin{cases}
\frac{e^2 n_{i}}{\epsilon _0}\left(\frac{r^2}{6}-\frac{R_{i}^2}{2}\right), &\text{для } r<R_{i}, \\
- \frac{e^2\,N_\mathrm{i}}{4\pi\epsilon_0\,r}, &r \geqslant R_i .  
\end{cases}
\end{equation}

Мы рассматривали сильно ионизованную водородоподобную наноплазму, с количеством электронов, равному количеству ионов $N_e=N_i=N$ в начальный момент времени в модели с точечными ионами и в модели с эффективным потенциалом. Начальная конфигурация кластера определялась количеством ионов $N_i$ и радиусом кластера $R_i$. Электронная подсистема генерировалась произвольным образом в сфере с радиусом $R_i$. В случае модели с эффективным потенциалом плотность ионов считалась однородной по всей сфере радиуса $R_i$ и находилась как $n_i = N_i/(4 \pi R_i)$. Во второй модели ионы также располагались произвольным образом, и плотность $n_i$  в этом случае представляла среднюю плотность по всему кластеру. Процесс взаимодействия электронов с излучением в явном виде не моделировался, а вводился в модель как начальная температура электронов $T$. 
В случае точечных ионов, электрон-ионное взаимодействие задавалась псевдопотенциалом (\ref{eq:md-perf}). Электрон-электронное взаимодействие задавалось потенциалом Кулона~(\ref{eq:md-pcoul}).

После задания начального положения, происходила термализация системы с помощью термостата Ланжевена. При достижении необходимой температуры $T$, термостат выключался и дальнейшие расчеты проводились без его использования. Подробнее процедура термализации описана в разделе~\ref{sec:cluster-termalization}. После выключения термостата, происходила ионизация кластера, до некоторого стационарного количества электронов в кластере (о методе исследования минимального заряда кластера будет написано в разделе~\ref{sec:cluster-charge}). В этом стационарном состоянии также рассчитывались автокорреляционные функции полного тока, которые использовались для расчета спектра электронных колебаний~\cite{RRRM-IJMPB08,Winkel-CPP13,Broda_CPP13}.

Свойства пространственно ограниченной плазмы сильно отличаются от неограниченной однородной плазмы из-за дальнодействующего характера кулоновского взаимодействия, которое приводит к возникновению значимых поверхностных эффектов в первом случае. Подобно случаю однородной плазмы~\cite{Selchow-PRE01}, электронные колебания можно анализировать, исходя из автокорреляционной функции полного тока (АКФТ).
Для локальной электронной плотности: 
\begin{equation}
	{\bf j}({\bf r},t)=\sum_{c=e,i}\sum_{\alpha}^N e_c{\bf v}_{c,\alpha}(t)
	\delta({\bf r}_{c,\alpha}(t)-{\bf r})
\end{equation}
автокорреляционная функция рассчитывается как:
\begin{equation}
\hat K_{jj}({\bf r},{\bf r}',\tau)=\lim_{T_\mathrm{avr} \to \infty} \frac{1}{T_\mathrm{avr}} \int_0^{T_\mathrm{avr}}  {\bf j}({\bf r},t){\bf j}({\bf r}',t+\tau)\, dt.
\end{equation}
Пространственное преобразование Фурье в этом случае записывается как:
\begin{equation}\label{current-acf}
K^{L/T}(\omega) = \frac{\omega^{}_\mathrm{pl}}{4\pi} \int K^{L/T}(t) \mathrm{e}^{i\omega t} dt,
\end{equation}
где $\omega^{}_\mathrm{pl} = \sqrt{e^2 n_\mathrm{e}/(\epsilon_0 m_e)}$ -- плазменная частота.

Пример Фурье-преобразования представлен на рисунках~\ref{fig:cluster-55spec} и~\ref{fig:cluster-SpectrEvolution}) для различных размеров кластеров при плотности ионов $n_i=2.7 \times 10^{22}\;\text{см}^{-3}$.
\begin{figure}[t!]
	\begin{multicols}{2}
		\centering
		
		\includegraphics[width=1\linewidth]{figure/Cluster-55-spectr.pdf}
		\caption{\label{fig:cluster-55spec}Фурье преобразование АКФТ (\ref{current-acf}) для различных моделей взаимодействия. Размер кластера $N_i=55$, начальная температура электронов $T_\mathrm{e}=2.2 \text{эВ}$.
		Штриховая линия отображает положение частоты~$\omega_{Mie}$.
		}
		\columnbreak

		\centering
		\label{fig:cluster-MiPosition}
		\includegraphics[width=1.025\linewidth]{figure/Cluster-freq-position.pdf}
		\caption{Положение резонансного пика соответствующего колебания Ми, в зависимости от размера кластера. Красные точки соответствуют результатам данной работы с $T_\mathrm{e}=2.2~\text{эВ}$, квадраты взяты из работы~\cite{bystryi2014electronic}, $T_\mathrm{e} = 1.0$~эВ, синие точки -- результаты для модели с эффективным потенциалом.}
	\end{multicols}
\end{figure}

Полученные результаты в приближении точечных ионов схожи с представленными в работах~\cite{RRRM-IJMPB08,MB-PRB15}. Спектры электронных колебаний представляют собой несколько пиков, с главным пиком близким к частоте Ми $\omega_\mathrm{Mie}^2=\omega_\mathrm{pl}^2/3$. В случае использования модели с эффективным потенциалом, спектр представляет собой узкую линию, на частоте, очень близкой к частоте Ми. Положение пика не зависит от размера системы, в отличии от случая с реальными ионами, где положение главного пика с увеличением размера системы приближалось к аналитическому значению частоты Ми (рис. \ref{fig:cluster-MiPosition}). Возможной причиной возникновения нескольких линий на спектре электронных колебаний и их сдвига с увеличением размера кластера, является неоднородность ионной подсистемы. При большом количестве ионов в кластере, ионная плотность становится более однородной, что приводит к уменьшению ширины пика и его сдвигу к частоте Ми.  Уменьшение амплитуды колебаний Ми и возникновение плазменных колебаний при увеличении размера кластера в текущих расчетах не наблюдалось, в отличие от работы~\cite{MB-PRB15}.

\begin{figure}[h!]
	\centering
	
	\includegraphics[width=0.5\linewidth]{figure/Cluster-Spectr-evolution.pdf}
	\caption{\label{fig:cluster-SpectrEvolution}Фурье преобразование АКФТ (\ref{current-acf}) для различных моделей взаимодействия и размеров кластера. Начальная температура электронов $T_\mathrm{e}=2.2 \text{эВ}$. Синяя линия отображает результаты для модели с эффективным потенциалом.
		Штриховая линия отображает положение частоты~$\omega_{Mie}$.
	}
\end{figure}

\section{Зависимость заряда и температуры кластера от начальных параметров}\label{sec:cluster-charge}
Методом классической МД, используя модели как с точечными ионами, так и с эффективным потенциалом, был изучен вылет электронов из кластера, который определяется размером кластера $R_i$, ионной плотность $n_i$ и электронной температурой $T_e$ в начальный момент времени. Так как движение электронов после выключения термостата не ограничивалось, то они могли, обладая достаточной энергией, вылетать из кластера. Вылет электрона из кластера приводит к увеличению общего заряда кластера $Z$, и как результат, к уменьшению вероятности последующего вылета электронов. 
С теоретической точки зрения, эмиссия электронов прекращается, когда кинетическая энергия всех электронов становится меньше, чем потенциал кластера. На практике наступление равновесного состояния можно определять по отсутствию изменения заряда кластера на протяжении достаточно длинного промежутка времени (наносекунды и больше).

Нами были рассмотрены кластеры различного размера с ионной плотностью $n_i=2.7 \times 10^{22}~\text{см}^{-3}$. Размер кластеров варьировался от 55 до $20 \times 10^{3}$ ионов.
Начальные  температуры электронов были равны $T_e~=~2.2~\text{эВ}$ и $~T_e~=~3.0~\text{эВ}$. Эти параметры соответствуют неидеальной плазме с параметром неидеальности $\Gamma = e^2/(4 \pi \epsilon_0 k_BT) (4 \pi n_\mathrm{e}/3)^{1/3}$ равном $3.17$ и $2.32$ соответственно (в начальный момент времени электронная плотность соответствует ионной $n_\mathrm{e} = n_\mathrm{i}$).
Финальный заряд кластера определялся как разность числа ионов и электронов в кластере $Z_\text{fin}~=~N_i~-~N_e$ после достижения равновесного состояния. 

Результаты МД моделирования представлены на рис. \ref{fig:cluster-Charge}. Как видно из рисунка, существует сильная зависимость заряда кластера от начальной температуры и размера кластера. Все полученные результаты могут быть аппроксимированы простым выражением~\cite{Broda_CPP13}:
\begin{equation}\label{eq:final_charge}
Z_\mathrm{fin} = c_1 R_\mathrm{i} T_\mathrm{fin}
\end{equation}
со значениями параметров $c_1=0.55 \pm 0.06~\mathrm{(a_B eV)}^{-1}$. Этот результат хорошо согласуется с работой~\cite{MB-PRB15}, и в меньшей степени с работой~\cite{Broda_CPP13}, где было получено значение  $c_1=0.65~\mathrm{(a_B eV)}^{-1}$. Также стоит отметить, что модели с реальными ионами и с эффективным потенциалом дают сходные результаты. 
\begin{figure}[t]
	\begin{multicols}{2}
		\centering
		
		\includegraphics[width=1\linewidth]{figure/Cluster-Zfinal.pdf}
		\caption{\label{fig:cluster-Charge}Конечный заряд кластера $Z_\text{fin}$ как функция от произведения  радиуса кластера $R_i$ на конечную температуру $T_\text{fin}$.
		Точки отображают результаты МД с различными моделям и начальной температуры $T_e$. Ромбы отображают результаты из~\protect\cite{Broda_CPP13}. 	
	}
	
		\centering
		
		\includegraphics[width=1\linewidth]{figure/Cluster-Tfinal.pdf}
		\caption{\label{fig:cluster-Temperature}Зависимость конечной температуры $T_\mathrm{fin}$ наноплазмы от начальной 
			температуры $T_\text{e}$ и радиуса кластера $R_\mathrm{i}$. Сравнение данных МД (квадратики и			
			крестики) с теоретической формулой – сплошные линии.}
	\end{multicols}
\end{figure}

Зависимость конечной температуры кластера (в квазиравновесном состоянии) от размера кластера, отображена на рисунке \ref{fig:cluster-Temperature}. Из рисунка видно, что относительные потери энергии в процессе ионизации уменьшаются с увеличением размера кластера. Для конечной температуры можно получить аналитическое выражение, рассматривая энергию, уносимую одной частицей в результате ее вылета:
\begin{equation}\label{eq:Cons-law}
\frac{d}{dt}\left(\frac{3}{2} N_\mathrm{e} k^{}_\mathrm{B} T\right) = - \frac{dZ}{dt} U_\mathrm{b}.
\end{equation}
Интегрируя данное уравнение от $T_0$ до $T_\text{fin}$ и от 0 до $Z$, получаем:
\begin{equation}\label{eq:main_after_integration}
k^{}_\mathrm{B} (T_\mathrm{fin} - T_0)
= -\frac{Z_\mathrm{fin}^2 e^2}{12\pi\epsilon_0\,R_\mathrm{i}N_\mathrm{e}},
\end{equation}
Исходя из этого выражения, можно записать выражение для конечной температуры:
\begin{equation}\label{eq:final_temperature}
T_\mathrm{fin}
= T_0 - \frac{e^2 c_1^2}{16 \pi^2 \epsilon_0  n_\mathrm{e}}\,
\frac{k^{}_\mathrm{B} T_\mathrm{fin}^2}{R_\mathrm{i}^2}
= T_0 - c_2 \left( \frac{T_\mathrm{fin}}{R_\mathrm{i}} \right)^2,
\end{equation}
где параметры $c_2 = e^2 c_1^2 k^{}_\mathrm{B}/(16 \pi^2 \epsilon_0  n_\mathrm{e})$. Для кластерной наноплазмы, рассмотренной в данной работе, параметр равен $c_2 = 165 \pm 35\; a_B^2/\mathrm{eV}$.Как видно из рис.~\ref{fig:cluster-Temperature}, теоретическая зависимость температуры от размера кластера ~(\ref{eq:final_temperature}) с хорошей точностью совпадает с результатами МД моделирования.

