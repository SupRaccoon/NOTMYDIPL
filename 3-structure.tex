\chapter{Экспериментальная часть}
\section{Параметры расчета}
Моделирование в данной  работе проводилось методом молекулярной динамики (МД). В качестве потенциала взаимодействия был взят потенциал погруженного атома с угловой зависимостью (ADP-потенциал \cite{mishin2005phase}) с параметризацией \cite{Smirnova}. Все расчеты были проведены  для кубической расчетной ячейки размером $21,5*21,5*21,5 \text{ нм}^3$  в NVE - ансамбле, для диапазона процентного содержания  Nb от 0 до 100 процентов и скорости охлаждения от 0,5 до 11 K/пс.  Для устранения поверхностных эффектов применялись периодические граничные условия. В расчетной ячейке поддерживалось постоянное давление посредством оставленного в ней пустого пространства, сплав перед началом плавления занимал объем $21,5*14,3*21,5 \text{ нм}^3$, температура структуры составляла $300 $ K, нагрев проводился до $2500$ K. Шаг интегрирования составлял $5 \cdot10^{-3}\text{ пс}$. Все расчеты проведены с использованием пакета LAMMPS \cite{Plimpton}.
\section{Анализ структуры}
\subsection{Анализ икосаэдрического окружения}
Анализ многогранников Вороного проводился с помощью программы OVITO \cite{Stukowski}. При анализе многогранников (0,12,0) отмечены следующие особенности:
\begin{enumerate}
	\item Такие многогранники  наблюдаются в аморфной фазе Zr-Nb, хотя и в достаточно небольшом количестве - максимальное процентное содержание порядка 6,8\% от общего числа многогранников.
	\item Их число резко возрастает при переходе «жидкость - стекло».
	\item Многогранники вида (0,12,0) образует практически полностью только Nb. Максимальное значение числа многогранников - порядка 14\% от числа атомов Nb. 
	\item Zr образует такие многогранники гораздо реже - в его случае максимальное число икосаэдрических кластеров не превышает 2\% процентов от общего числа атомов Zr.
\end{enumerate}

Для объяснения того, что икосаэдрические кластеры почти полностью образуются только вокруг Nb, был использован стерический фактор, которым объяснялось аналогичное поведение компонентов аморфного сплава в работах  \cite{Sheng},\cite{Fukunaga}, \cite{Pryadilschikov} и тд.  Важное влияние оказывает относительный размер атомов, который в случае металлов играет роль длины связи. Дело в том, что чем больше разница в размерах атомов, содержащихся в сплаве, тем больше соседей возможно разместить вокруг атома. Соответственно, чем больше один из атомов по сравнению со вторым, тем больше будет соседей у него, и, соответственно, тем меньше соседей  у атома с меньшим радиусом. Из работы \cite{Tretyakov}  радиус Zr равен 160 пм, Nb - 145 пм. Соответственно,  радиус атома Zr больше, и вокруг него расположено более чем 12 атомов. К аналогичным выводам пришли для другого аморфного бинарного сплава в работе \cite{Bondarev}.

Для проверки данной гипотезы было проанализировано распределение атомов по числу соседей. Оказалось, что у Nb в среднем порядка 12-13 соседей (более трети атомов с 12 соседями), а в случае Zr менее 8 \% атомов имеет 12 соседей - больше всего атомов с 14-15 соседями (рис. \ref{chislo_sosedey}). 
\begin{figure}[h!]
	\begin{center}
		\includegraphics[scale= 0.5]{fig5}{}{}{}
		\caption{Распределение атомов по числу соседей  для Zr и Nb (40\% Nb).}
		\label{chislo_sosedey}
	\end{center}
\end{figure}

Также для сравнения были рассмотрены аморфные сплавы Ni-Nb и Ni-Zr. Моделирование было проведено для тех же условий с помощью EAM-потенциалов  \cite{Zhang}, \cite{Mendelev}. Радиус Ni 124 пм \cite{Tretyakov}, и в этих структурах он являлся меньшим. После проверки распределения атомов по соседям для структуры NiNb  получено (рис. \ref{Ni}), что  у Ni-Nb пики для Ni на 12 и 13 соседях (более 36\% атомов Ni с 12 соседями), для Nb - на 15-16 (более трети атомов с 15 соседями). В результате в сплаве Ni-Nb  уже Nb не образует икосаэдрических кластеров, при это более 12\% многогранников Вороного, построенных вокруг атомов Ni, относятся к виду (0,12,0). Аналогичная картина наблюдается в случае  Ni-Zr  - у Zr в среднем порядка 15-16 соседей и он не образует икосаэдры. У Ni - 12-13 соседей, при этом порядка 7\% из них образуют икосаэдрические кластеры. Таким образом, можно сделать вывод, что отсутствие икосаэдрических кластеров, образованных атомами Zr вызвано стерическими факторами.
\begin{figure}[h!]
	\begin{minipage}[h]{0.5\linewidth}
		\center{\includegraphics[width=1\linewidth]{fig6}} a \\
	\end{minipage}
	\hfill
	\begin{minipage}[h]{0.5\linewidth}
		\center{\includegraphics[width=1\linewidth]{fig7}} \\б
	\end{minipage}
	\vfill
	\begin{minipage}[h]{0.5\linewidth}
		\center{\includegraphics[width=1\linewidth]{fig8}} в \\
	\end{minipage}
	\hfill
	\begin{minipage}[h]{0.5\linewidth}
		\center{\includegraphics[width=1\linewidth]{fig9}} г \\
	\end{minipage}
	\caption{Распределение по числу соседей в сплавах Ni-Zr и Ni-Nb (сверху) и соответствующие им ПКФ (снизу) (сплав Ni-Nb с 30\% Ni, Ni-Zr с 30\% Zr).}
	\label{Ni}
\end{figure}

Для многогранников (0,12,0) была изучена зависимость их числа  от скорости охлаждения  и процентного содержания Nb (рис. \ref{Vp-p}). Первая зависимость  монотонная - процентное содержание таких многогранников снижается с ростом скорости охлаждения. Это объясняется с учетом кинетического характера аморфизации - структура не успевает перестроится из-за высокой скорости охлаждения. Зависимость числа данных многогранников от процентного содержания Nb немонотонна. В начале наблюдается рост, пик в районе 35\% , после чего линейное падение. Это можно объяснить следующим образом. В начале преобладает тот факт, что чем больше Nb, тем «аморфнее» соединение (это рассмотрено в пункте 6), поэтому многогранников больше. Но, чем больше Nb, тем в среднем у Nb больше соседей. Поэтому постепенно сокращается количество кластеров, соответствующих 12 соседям, в частности (0,12,0). 
\begin{figure}[h!]
	\begin{center}
		\includegraphics[scale= 0.65]{fig10}{}{}{}
		\caption{Зависимость числа многогранников (0,12,0) от процентного содержания Nb для различных скоростей охлаждения.}
		\label{Vp-p}
	\end{center}
\end{figure}

Для проверки данного предположения были  рассмотрены зависимости числа атомов Nb с 12, 13 и  14 соседями от процентного содержания Nb. Они описываются линейной зависимостью, с 12 атомами - убывающей, с 14 атомами - возрастающей. При этом общее число атомов с 12 или 14 соседями почти постоянное и не зависит от процентного содержания Nb (рис. \ref{sosedi_pr}). Число атомов Nb с 13 соседями также почти постоянное. Это согласуется с предположением о росте числа соседей из-за стерического фактора и перестроением икосаэрических кластеров в кластеры с 14 соседями.
\begin{figure}[h!]
	\begin{center}
		\includegraphics[scale= 0.6]{fig11}{}{}{}
		\caption{Зависимость числа атомов с 12, 13 и  14 соседями, cуммарного числа атомов с 12 и 14 соседями от процентного содержания Nb.}
		\label{sosedi_pr}
	\end{center}
\end{figure}

Подобные переходы широко обсуждаются в литературе. Икосаэдрические кластеры перестраиваются и видоизменяются с целью увеличения плотности и занятого объема, поскольку это выгодно по энергии \cite{Sheng}.  Также эти переходы  подтверждаются  зависимостями различных многогранников Вороного от температуры (с 12, 13 и 14 гранями). При росте процентного содержания увеличивается число многогранников с 13-14 гранями, а с 12, наоборот, уменьшается. 

Наиболее явно эти переходы отражены в зависимостях (0,12,0) и (2,8,4) от процентного содержания Nb. Эти зависимости очень схожи с зависимостями числа атомов с 12 и 14 соседями соответственно. 
Таким образом, действительно можно говорить о том, что происходит переход между этими 2 типами многогранников, и постепенно икосаэдрические кластеры заменяются на кластеры, соответствующие многогранникам вида (2,8,4). При увеличении скорости охлаждения повышается число атомов Nb с 12 и 13 соседями, что объясняется также кинетическим характером аморфизации и тем, что структура не успевает перестроиться в более стабильную. 

Также были проанализированы зависимости еще нескольких основных многогранников, образуемых Nb, от процентного содержания Nb (рис. \ref{polyhedra_Zr_Nb}б). При увеличении процентного содержания Nb наблюдается рост многогранников вида (3,6,4) и (1,10,2). Помимо того, что это можно объяснить увеличением числа соседей для атомов Nb, это связано c тем, что эти многогранники образуются из многогранников (0,12,0) небольшим количеством перестроений. Таким образом, эти многогранники тоже соответствуют икосаэдрическому окружению, а также наиболее вероятному числу соседей при таком соотношении атомных радиусов. В случае икосаэдров с центром в Zr (рис. \ref{polyhedra_Zr_Nb}а) общая тенденция заключается в том, что почти все кривые меняются очень плавно, не наблюдается сильного роста, как в случае с многогранниками, образованными Nb. Стоит также отметить рост числа многогранников вида (2,8,5) и (2,8,4) отвечающих большому числу соседей, а также сокращение числа многогранников вида (3,6,4), объясняемое снижением числа атомов с 13 соседями. Результаты, полученные в данном исследовании по структуре сплава для различного процентного содержания Nb находятся в хорошем согласии с работой \cite{Reddy}.
\begin{figure}[h!]
	\begin{center}
		\begin{minipage}[h]{0.7\linewidth}
			\center{\includegraphics[width=1\linewidth]{fig12}} \\a 
		\end{minipage}
		\hfill
		\begin{minipage}[h]{0.7\linewidth}
			\center{\includegraphics[width=1\linewidth]{fig13}} \\б
		\end{minipage}
		\caption{Зависимости числа различных многогранников Вороного, образованных  a) Zr и б) Nb соответственно, от процентного содержания Nb.}
		\label{polyhedra_Zr_Nb}
	\end{center}
\end{figure}

\subsection{Анализ ПКФ Zr-Nb, связь ПКФ и икосаэдрического окружения}
В работе проанализированы  три типа ПКФ - Zr-Zr, Zr-Nb и Nb-Nb (рис. \ref{RDF_full}). Можно отметить, что на ПКФ Zr-Zr (рис. \ref{RDF_full}а) отсутствует расщепление второго пика. Вместо него наблюдается плавно спадающий второй пик. Также между первым и вторым пиком выделяется достаточно высокий подпик, не характерный для ПКФ аморфных сплавов. При этом перекрестная ПКФ Zr-Nb (рис. \ref{RDF_full}б)  выглядит типично для аморфных соединений - наблюдается характерное расщепление второго пика ПКФ.
У ПКФ Nb-Nb (рис. \ref{RDF_full}в)  также существуют некоторые особенности - вместо цельного второго пика наблюдается подпик, широкое плато после него и очень слабый второй подпик. 

\begin{figure}[h!]
	\begin{minipage}[h]{0.32\linewidth}
		\center{\includegraphics[width=1\linewidth]{fig14}} a \\
	\end{minipage}
	\hfill
	\begin{minipage}[h]{0.32\linewidth}
		\center{\includegraphics[width=1\linewidth]{fig15}} \\б
	\end{minipage}
	\hfill
	\begin{minipage}[h]{0.32\linewidth}
		\center{\includegraphics[width=1\linewidth]{fig16}} в \\
	\end{minipage}
	\caption{ПКФ а) Zr-Zr, б) Zr-Nb и в) Nb-Nb (50\% Nb).}
	\label{RDF_full}
\end{figure}


Таким образом, на ПКФ Zr-Zr и Nb-Nb в аморфном сплаве Zr-Nb не наблюдается обычное  для аморфных металлов расщепление второго пика. При этом нужно отметить, что расщепление второго пика ПКФ  Zr-Nb действительно является критерием аморфизации, поскольку оно наблюдается только в аморфной фазе. 

В случае ПКФ Ni-Nb картина выглядит похожим образом  - расщепление второго пика наблюдается на ПКФ Ni-Nb, а также на ПКФ Ni-Nb, на ПКФ Nb-Nb оно отсутствует (рис. \ref{Ni}в). 
ПКФ для случая сплава Ni-Zr имеет достаточно нехарактерный вид, что  может быть обусловлено большой разницей в радиусах атомов (рис. \ref{Ni}г) . При этом, в обоих этих сплавах именно атом Ni образует икосаэдрическое окружение, из чего можно сделать предположение о существовании связи между икосаэдрическим окружением атомов и расщеплением второго пика ПКФ

Для нахождения связи между этими двумя критериями из расчетной ячейки были выделены атомы с икосаэдрическим окружением, то есть атомы с 12 ближайшими атомами,  и соседние с ними атомы. Для получившегося набора частиц можно отметить, что количество атомов, задействованных в образовании икосаэдрической структуры велико - более 45\% атомов (c учетом того, что икосаэдрическое окружение характерно только для 6\% атомов и что в рассматривались только многогранники вида (0,12,0) без 
учета деформированных икосаэдров). Следовательно, в образовании икосаэдров задействована почти половина атомов вещества. Также  эти икосаэдры образуют основу структуры аморфного соединения. Они равномерно распространены по всей расчетной ячейке, а атомы, не являющиеся частью икосаэдров, заполняют оставшиеся в икосаэдрической структуре полости. Заполнения всей структуры данными кластерами не происходит, поскольку икосаэдрическая упаковка не является плотнейшей. Поэтому икосаэдрическая структура не повторяется  и не заполняет все пространство \cite{Hoare}. Также в структуре  практически отсутствуют изолированные икосаэдры. Все икосаэдры связаны друг с другом, причем практически всегда кластеры взаимопроникающие, то есть связь очень сильная - по средством семи обобщенных атомов (рис. \ref{icosahedra_ex}б). В этом случае необходимо разорвать 15 связей для отделения икосаэдра от всей остальной  структуры. Преобладание именно такого типа связи подтверждается следующими фактами:
\begin{enumerate}
	\item В структуре с 40\%  содержанием Nb отношение атомов, задействованных в икосаэдрах к атомам с икосаэдрическим окружением  $N_{at}/N_{ico}\approx 7,8$;
	\item В изолированном кластере $N_{at}/N_{ico}=13$  (вершины кластера и его центр);
	\item Для структуры, состоящей только из взаимопроникающих кластеров, справедливо соотношение $N_{at} = 6*N_{ico}+7$. То есть при большом числе икосаэдров в такой структуре отношение атомов,  задействованных в икосаэдрах, к атомам с икосаэдрическим окружением стремится к 6;
	\item В случае связи по вершинам  $N_{at}/N_{ico}=12$, связи по ребру - $11$, связи по грани - $10$. 
\end{enumerate}

Таким образом, полученная  структура образована наиболее крепкими типами связи - каждый  икосаэдрический кластер сильно связан с икосаэдрической структурой в целом. Также нужно отметить, что атомы Zr и Nb  в равной степени участвуют в образовании икосаэдрического окружения - в выделенной структуре процентное содержание атомов  Zr и Nb практически такое же, как и в изначальной - 59\% Zr и 41\% Nb.

Отметим основные особенности ПКФ  икосаэдрической структуры ($\text{ПКФ}_{ico}$), которая обладает рядом отличий по сравнению с ПКФ структуры в целом (рис. \ref{RDF_full}, рис. \ref{RDF_ico_besico}). Для нее характерно  расщепление второго пика ($\text{П}_2$) на ПКФ Nb-Nb.  При этом, на ПКФ атомов, не задействованных в икосаэдрах ($\text{ПКФ}_{ooico}$),  характерное расщепление $\text{П}_2$ на ПКФ Zr-Nb (рис. \ref{RDF_ico_besico}г) отсутствует, а на ПКФ Nb-Nb (рис. \ref{RDF_ico_besico}е)   выделяется широкий второй пик. При этом,  ПКФ Zr-Zr больше соответствует стандартному виду ПКФ аморфных металлов с тем  отличием, что вместо второго подпика второго пика ($\text{П}_2^2$)  выделяется широкое плато после первого подпика второго пика ($\text{П}_2^1$) .На ПКФ Zr-Zr (рис. \ref{RDF_ico_besico}а) $\text{П}_2$ значительно уже, чем на ПКФ Zr-Zr всех атомов аморфной структуры (рис. \ref{RDF_full}а) и на ПКФ атомов, лежащих вне икосаэдрических кластеров (рис. \ref{RDF_ico_besico}б). Наконец, если сравнить полные ПКФ атомов икосаэдрической структуры и всех атомов аморфного сплава (рис. \ref{RDF_full_ico}), можно заметить, что расщепление второго пика наблюдается значительно сильнее в случае атомов икосаэдрической структуры. Таким образом, можно сделать вывод, что расщепление $\text{П}_2$, которое наблюдается  на ПКФ, обусловлено именно образовавшейся в веществе икосаэдрической структурой. 
\begin{figure}[h!]
	\begin{minipage}[h!]{0.47\linewidth}
		\center{\includegraphics[width=1\linewidth]{fig17}} a \\
	\end{minipage}
	\hfill
	\begin{minipage}[h!]{0.47\linewidth}
		\center{\includegraphics[width=1\linewidth]{fig18}} б \\
	\end{minipage}
	\vfill
	\begin{minipage}[h!]{0.47\linewidth}
		\center{\includegraphics[width=1\linewidth]{fig19}} \\в
	\end{minipage}
	\hfill
	\begin{minipage}[h!]{0.47\linewidth}
		\center{\includegraphics[width=1\linewidth]{fig20}} г \\
	\end{minipage}
	\vfill
	\begin{minipage}[h!]{0.47\linewidth}
		\center{\includegraphics[width=1\linewidth]{fig21}} \\д
	\end{minipage}
	\hfill
	\begin{minipage}[h!]{0.47\linewidth}
		\center{\includegraphics[width=1\linewidth]{fig22}} е \\
	\end{minipage}
	\caption{ПКФ Zr-Zr, Zr-Nb и Nb-Nb для атомов, содержащихся в кластерах (а), в) и д), соответственно), и атомов вне кластеров (б),г) и е), соответственно).}
	\label{RDF_ico_besico}
\end{figure}

\begin{figure}[h!]
	\begin{center}
		\includegraphics[scale= 0.6]{fig23}{}{}{}
		\caption{Полная ПКФ атомов в икосаэдрических кластеров и атомов всей аморфной структуры.}
		\label{RDF_full_ico}
	\end{center}
\end{figure}
Для объяснения полученной $\text{ПКФ}_{ico}$ были проанализированы отдельные икосаэдры и взаимопроникающие кластеры.  Геометрическое рассмотрение идеального икосаэдрического кластера и двух взаимопроникающих кластеров приведено в  работе \cite{Liang}. Расстояния, которые лежат в области первого пика ПКФ ($\text{П}_1$), соответствуют расстоянию между ближайшими соседями, $\text{П}_2^1$ соответствует расстояние между двумя атомами в одном пятиугольнике, разделенные третьим атомом (рис. \ref{icosahedra_ex}а, атомы 3 и 4). В случае $\text{П}_2^2$ уже нельзя выбрать конкретные атомы, соответствующие ему, поскольку на относительное расположение атомов при таких расстояниях сильно влияет большое количество атомов окружения. Рассмотрим более подробно кластеры, приводя для каждого интересующего отрезка ПКФ атомы, расстояние между которыми соответствуют этому отрезку. 
\begin{figure}[h!]
	\begin{minipage}[h]{0.5\linewidth}
		\center{\includegraphics[width=1\linewidth]{fig3}} a \\
	\end{minipage}
	\hfill
	\begin{minipage}[h]{0.5\linewidth}
		\center{\includegraphics[width=1\linewidth]{fig4}} б\\
	\end{minipage}
	\caption{ Пример а)  икосаэдрического кластера; б) двух взаимопроникающих кластеров (серые - атомы Nb, черные - Zr).}
	\label{icosahedra_ex}
\end{figure}
В начале была рассмотрена ПКФ Zr-Zr. Первый  участок - высокая ступень после $\text{П}_1$. Этой ступени соответствует расстояние между вершиной икосаэдра и атомом, лежащим в ближайшем пятиугольнике (рис. \ref{icosahedra_ex}а, 1 и 5 атомы). В случае ПКФ Nb-Nb аналогичное расстояние между атомами дает вершину, лежащую внутри $\text{П}_1$. Объяснением этому служит стерический фактор. К примеру, в икосаэдре, изображенном на рис.\ref{icosahedra_ex} а видно, что эта связь находится в икосаэдре  с еще одним атомом Zr, имеющем связи с обоими отмеченными атомами. В результате этого связь становится длиннее, чем в случае ПКФ Nb-Nb.  Аналогичные комбинации были обнаружены еще в нескольких кластерах, там они тоже дают ступень после $\text{П}_1$. 

После этого было исследовано, с чем связано отсутствие расщепления $\text{П}_2$ Zr-Zr  $\text{ПКФ}_{ico}$ .  Первое, что нужно отметить - $\text{ПКФ}_{ico}$ Zr-Zr очень  плавно уменьшается после второго пика и там, где в случае Nb различим отдельный $\text{П}_2^2$, наблюдается широкое плато. Для интерпретации этого участка $\text{ПКФ}_{ico}$ была рассмотрена система взаимопроникающих кластеров. Такая система представлена на рис. \ref{icosahedra_ex}б. На ней выделены два атома, расстояние между которыми соответствует рассматриваемому участку -  расстояние между атомами, лежащими в пятиугольниках $C$ и $E$ через один пятиугольник $D$ (рис. \ref{RDF_ico_besico}а, рис. \ref{icosahedra_ex}б 6 и 7 атомы). При этом данное расстояние определяется очень большим количеством атомов, что и может в результате давать широкий разброс по расстоянию, и, как следствие, широкое плато вместо одиночного пика. Еще одна пара атомов, расстояние между которыми соответствует плато после $\text{П}_2$, это расстояние между вершинами пятиугольников $A$ и $B$, соединяющий вектор которых проходит через центр пятиугольника (рис. \ref{RDF_ico_besico}а, рис. \ref{icosahedra_ex}а атомы 3 и 5). Также существует еще одна комбинация - вершина икосаэдра и атом во втором от него пятиугольнике $D$ (рис. \ref{RDF_ico_besico}а, рис. \ref{icosahedra_ex}б 8 и 9 атомы). В случае увеличения размера икосаэдров в сплаве с 30\% Nb плато после $\text{П}_2^1$ соответствует расстоянию между двумя атомами в  одном пятиугольнике $B$ через один промежуточный (рис.\ref{RDF_ico_besico}а, рис. \ref{icosahedra_ex}а 3 и 4 атомы).  

Последний рассматриваемый участок $\text{ПКФ}_{ico}$ Zr-Zr - $\text{П}_2^1$. По теоретическим расчетам, проведенным для правильного икосаэдра, этот пик соответствует расстоянию между двумя атомами в пятиугольнике, разделенными промежуточным атомом. Это действительно наблюдается  (рис.\ref{RDF_ico_besico}а, рис.\ref{icosahedra_ex}а 3 и 4 атомы). На практике, был обнаружен еще один возможный вариант - два атома в соседних пятиугольниках $D$ и $E$ , повернутые на один атом друг относительно друга (рис.\ref{RDF_ico_besico}а, рис. \ref{icosahedra_ex}б 7 и 9 атомы).

Также данный подход был использован для объяснения $\text{ПКФ}_{ico}$ Nb-Nb. Некоторые пики соответствуют тем же расстояниям что и раньше -  первый пик $\text{ПКФ}_{ico}$ образован расстоянием между ближайшими соседями (рис.\ref{RDF_ico_besico}д, рис.\ref{icosahedra_ex} 6 и 12 атомы). $\text{П}_2^1$ $\text{ПКФ}_{ico}$ образован расстоянием между двумя атомами в пятиугольнике, разделенными промежуточным атомом (рис. \ref{RDF_ico_besico}д, рис. \ref{icosahedra_ex}а 3 и 4 атомы). Для исследования расстояний, которые соответствуют  $\text{П}_2^2$, была рассмотрена система из нескольких икосаэдров. Этому подпику соответствует расстояние между  атомами в пятиугольниках, разделенных одним пятиугольником (рис. \ref{RDF_ico_besico}д, рис. \ref{icosahedra_ex}б 7 и 12 атомы, пятиугольники $C$ и $E$). Возможна также еще одна комбинация, аналогичная предыдущей - тоже два атома в пятиугольниках, разделенных одним пятиугольником,  только атомы уже не ближайшие, а повернутые на один друг относительно друга (рис. \ref{RDF_ico_besico}д, рис. \ref{icosahedra_ex}б  6 и 7 атомы в пятиугольниках $C$ и $E$). Наконец, последний вариант - расстояние между центром одного икосаэдра и самой дальней вершиной второго икосаэдра (по сути, противоположные вершины одного икосаэдра (рис. \ref{RDF_ico_besico}д, рис. \ref{icosahedra_ex}б 10 и 11 атомы). 


Как можно заметить, количество возможных вариантов больше, чем в случае $\text{ПКФ}_{ico}$ Zr-Zr за счет атомов, лежащих в центре икосаэдров. Поэтому и $\text{П}_2^2$ в случае  Nb-Nb явно выражен, в отличие от $\text{ПКФ}_{ico}$ Zr-Zr - он реализуется большим числом способов, и, соответственно, чаще встречается в структуре. Для $\text{ПКФ}_{ico}$ Zr-Nb все пики соответствуют разобранным выше комбинациям атомов.


Таким образом, в работе полностью был объяснен вид всех  ПКФ - Zr-Zr, Nb-Nb и Zr-Nb. В частности, было показано, что расщепление второго пика ПКФ вызвано  икосаэдрической подструктурой, образующейся в аморфном сплаве. Эта структура состоит из взаимопроникающих кластеров, заполняющих весь объем вещества.

\section{Температура стеклования}
В пункте 2 были описаны косвенные критерии стеклования, использованные в работе. Определяемая с их помощью температура стеклования зависит от того, какие изменения берутся за основу. К примеру, в экспериментах зачастую используют изменения коэффициентов теплоемкости \cite{tropin2015heat} и вязкости \cite{konstantinova2009kinematicheskaya}, а также изменение коэффициента Холла \cite{Kuzmenko}. Далее рассмотрены структурные критерии стеклования,  результаты определения температуры стеклования и их сравнение друг с другом.

\subsection{Зависимость числа икосаэдрических кластеров от температуры}
Для сравнения были рассмотрены  зависимости  числа икосаэдрических кластеров от температуры при охлаждении расплава и при плавлении аморфной фазы, а также зависимости числа других многогранников для тех же процессов. На  рис.\ref{0120_284}  приведены зависимости многогранников вида (0,12,0) и  (2,8,4) в качестве примера. Для многогранников (2,8,4) можно отметить, что данные, полученные при нагревании и на охлаждении практически идентичны, процессы проходили по одному пути. Также на графике отсутствуют какие-либо резкие перегибы и переломы на зависимости числа многогранников от температуры, то есть зависимости числа таких многогранников в жидкости и в аморфной фазе одинаковы и разграничить эти фазы по данной зависимости не представляется возможным.
\begin{figure}[h!]
	\begin{center}
		\begin{minipage}[h]{0.6\linewidth}
			\center{\includegraphics[width=1\linewidth]{fig24}} \\a 
		\end{minipage}
		\hfill
		\begin{minipage}[h]{0.6\linewidth}
			\center{\includegraphics[width=1\linewidth]{fig25}} \\б
		\end{minipage}
		\caption{а) Зависимость числа многогранников (2,8,4)  от температуры при охлаждении расплава Zr-Nb и плавлении аморфной фазы Zr-Nb, 55\% Nb. 
			б)Зависимость числа многогранников (0,12,0) от температуры при охлаждении расплава Zr-Nb и плавлении аморфной фазы Zr-Nb, 55\% Nb. Сплошная кривая - аппроксимация функции $N_{ic}(T)$ при плавлении сигмоидальной функцией. На вставке - вторая производная функции количества многогранников (0,12,0) от температуры.}
		\label{0120_284}
	\end{center}
\end{figure}
При этом на зависимости числа икосаэдрических кластеров от температуры наблюдается гистерезис, причем он тем сильнее выражен, чем глубже находится рассматриваемый сплав в области существования аморфной фазы на фазовой диаграмме. Также на зависимости, соответствующей нагреванию аморфного сплава, можно отметить три характерных участка.

При  увеличении  температуры сначала наблюдается участок, соответствующий аморфной фазе, после этого участок, на котором происходит  резкое уменьшение числа икосаэдрических кластеров, а в конце  происходит переход в область жидкости, где число икосаэдров при росте температуры меняется уже не так резко (снижается чуть более чем на процент при изменении температуры на $\approx 600\text{ K}$). Таким образом, возможно выделить  диапазон температур, соответствующий переходной фазе между жидкостью и стеклом. В этом переходном состоянии идет разрушение икосаэдрических кластеров, и, как следствие, разрушение аморфной структуры вещества.

На приведенном рис. \ref{0120_284}б  показаны рассчитанные значения числа  икосаэдров при различных температурах. Данная кривая лучше всего аппроксимируется сигмоидальной зависимостью:
\begin{equation}
N_{ico} = \frac{A}{1+e^{-k(T-b)}}
\label{eq_3}
\end{equation} 
Для нахождения точек изменения зависимости икосаэдров от температуры были найдены точки экстремума второй производной аппроксимирующей функции:
\begin{equation}
\frac{d^2N_{ico}}{dT^2} = \frac{2Ak^2e^{-2\cdot k(T-b)}}{(1 + e^{-k(T-b)})^3}-\frac{k^2e^{-k\cdot(T-b)}}{(1 + e^{-k(T-b)})^2 }
\label{eq_4}
\end{equation} 
Полученные температуры соответствуют температурам $T_1$ и $T_2$, разделяющих график на три характерных участка.  Также по графику второй производной можно отметить, что на промежутке температур $T_1$ - $T_2$ скорость изменения числа многогранников максимальна.
Необходимо сказать, что диапазон температур, отвечающих переходному состоянию, очень широк - от  $1100 K$ до $1850 K$. Температурой стеклования выбрана температура, при которой начинается плавление аморфной фазы, поскольку именно при данной температуре исчезает расщепление второго пика, являющееся критерием аморфной фазы.  Наконец, нужно отметить, что определить температуры изменения зависимостей можно только на плавлении стекла. При охлаждении не наблюдается перелома, соответствующего стеклованию сплава (рис. \ref{0120_284}б,  Табл. \ref{tab:my-table}). Данная особенность, наблюдаемая при исследовании перехода  <<жидкость-стекло>>  с помощью различных структурных и термодинамических характеристик, широко рассматривается в литературе. Теоретически данная особенность объясняется в \cite{tropin2015heat}, \cite{tropin2011dependence}  с помощью параметра структурного порядка. В экспериментальных работах (к примеру, \cite{aji2015kinetic}) явление гистерезиса в переходах <<жидкость-стекло>>  и  <<стекло-жидкость>>  объясняется следующим образом.  Переход из стабильного состояния (жидкости) в метастабильное состояние (стекло) при температурах, меньших чем температура стеклования, происходит самопроизвольно, то есть при фиксации температуры ниже $T_g$ процесс образования аморфной фазы продолжается. Переход из стекла в жидкость в то же время гораздо больше похож на фазовые переходы второго рода \cite{Dandar}, процесс не идет самопроизвольно.

\subsection{Критерий Вендта-Абрахама}
Характерный вид зависимости отношения первого минимума к первому максимуму $g_{min}/g_{max}$ ПКФ от температуры, с помощью которого определялась температура стеклования по критерию Вендта-Абрахама, показан на рис. \ref{VA_criterium}.  
\begin{figure}[h!]
	\begin{center}
		\includegraphics[scale= 0.6]{fig2}{}{}{}
		\caption{Зависимость отношения первого минимума к первому максимуму ПКФ от температуры, 55\% Nb.}
		\label{VA_criterium}
	\end{center}
\end{figure}
Основные особенности данной зависимости заключаются в следующем. Во-первых, наблюдается гистерезис - различие кривых  охлаждения жидкости и плавления аморфного сплава. Это является еще одним доказательством связи вида ПКФ и икосаэдрического окружения атомов - подобное различие поведения числа многогранников от температуры между охлаждением и нагреванием наблюдается исключительно для многогранников (0,12,0), а значит, именно ими и определяется вид ПКФ. Во-вторых, определяемая в данном случае температура очень сильно зависит от того, каким образом решено разбить график на две прямых. Поэтому точность данного метода не является  высокой. Возвращаясь к предыдущему подпункту, можно сказать, что температура стеклования, определенная по данному критерию, соответствует  температуре, разделяющей аморфную и переходные фазы (Табл. \ref{tab:my-table}).

\subsection{Площадь под вторым пиком ПКФ}
Характерный вид зависимости площади под двумя разделенными подпиками второго пика ПКФ представлен на рис. \ref{Square_criterium}.
\begin{figure}[h!]
	\begin{center}
		\includegraphics[scale= 0.6]{fig26}{}{}{}
		\caption{Зависимость площади под вторым пиком ПКФ Zr-Nb от температуры, 50\% Nb. ($S_{max}$  соответствует наибольшей площади под вторым пиком ПКФ).}
		\label{Square_criterium}
	\end{center}
\end{figure}

На ней наблюдается, как и в случаях с зависимостью числа икосаэдрических кластеров от температуры и критерия Вендта-Абрахама, гистерезис - при охлаждении и нагревании кривые различаются, как и температуры, определяемые этим методом. При этом для данного критерия скачок площади, соответствующий температуре стеклования, явный. Поэтому температура определяется гораздо точнее, чем в двух предыдущих методах. В целом, этот метод показал хорошее согласие с двумя другими - температура стеклования, определяемая по зависимости площади от температуры, совпадает с двумя другими структурными методами (Табл. \ref{tab:my-table}). Особым достоинством данного метода является возможность определения температуры стеклования при охлаждении,  в отличие от двух других структурных методов. Тем не менее, нужно отметить, что для исключения влияния погрешностей расчета ПКФ на точность определения температуры стеклования, необходимо использовать систему с большим количеством атомов в расчетной ячейке (примерно $10^5$). 


\begin{table}[h!]
	\centering
	\caption{Таблица температур стеклования для скорости $11\cdot10^{12}  \text{ K/пс}$.}
	\label{tab:my-table}
	\resizebox{\textwidth}{!}{%
		\begin{tabular}{|c|c|c|c|c|c|c|}
			\hline
			& \multicolumn{2}{|c|}{$T$ \textit {по площади под вторым пиком,} $K$,} & $T$ \textit{по критерию Вендта-Абрахама,} $K$, & $T$ \textit{по икосаэдрическим кластерам,}$K$,  \\ 
			& \multicolumn{2}{|c|} {$\Delta T = 10K$}& \multicolumn{1}{|c|} {$\Delta T = 50K$} & \multicolumn{1}{|c|} {$\Delta T = 50K$} \\ \hline
			\%, Nb & охлаждение & нагревание  & \multicolumn{1}{|c|}{нагревание} & \multicolumn{1}{|c|}{нагревание}\\ \hline
			25 & 1100 & 1050  & 1000 & 1000 \\ \hline
			27,5 & 1100 & 1050  & 1000 & 1000 \\ \hline
			30 & 1050 & 1100  & 1200 & 1100 \\ \hline
			35 & 1000 & 1100  & 1150 & 1150 \\ \hline
			40 & 1200 & 1250  & 1200 & 1200 \\ \hline
			45 & 1150 & 1050  & 1200 & 1100 \\ \hline
			50 & 1100 & 1050  & 1200 & 1200 \\ \hline
			55 & 1000 & 1100  & 1200 & 1200 \\ \hline
			65 & 1000 & 1100  & 1100 & 1100 \\ \hline
			75 & 900 & 800  & 1000 & 1000 \\ \hline
		\end{tabular}%
	}
\end{table}
\section{Фазовая диаграмма}
Проведенные расчеты по стеклованию расплава Zr-Nb для различного процентного содержания и различной скорости охлаждения позволили получить фазовую диаграмму состояний Zr-Nb в координатах «Процентное содержание Nb - скорость охлаждения» (рис. \ref{K(T)}). На ней выделены три области, соответствующие трем различным состояниям. Центральная область соответствует аморфному сплаву, внешняя область - нанокристаллическому соединению, между ними - переходное состояние, в котором уже наблюдаются зародыши кристаллов, но структура вещества в целом еще аморфна. Минимальная скорость, при которой является возможным получить аморфный сплав Zr-Nb заданного процентного содержания, называется критической скоростью охлаждения $K$. 
\begin{figure}[h!]
	\begin{center}
		\includegraphics[scale= 0.15]{fig28}{}{}{}
		\caption{Зависимость критической скорости охлаждения Zr-Nb от процентного содержания Nb.}
		\label{K(T)}
	\end{center}
\end{figure}
Важным выводом из полученной фазовой диаграммы  является то, что минимум критической скорости не соответствует максимальному процентному содержанию икосаэдрических кластеров, которые являются основными структурными единицами металлического стекла. Для зависимости процентного содержания икосаэдрических кластеров от процентного содержания Nb максимум соответствует примерно 35\% Nb, а для критической скорости минимум приходится на 45-50\% Nb. 

\begin{figure}[h!]
	\begin{center}
		\includegraphics[scale= 0.3]{fig27}{}{}{}
		\caption{Зависимости отношения числа атомов в икосаэдрических кластерах в сплаве к числу икосаэдрических  кластеров от процентного содержания Nb и числа кластеров от процентного содержания Nb.}
		\label{icocenterofico}
	\end{center}
\end{figure}
Возможным объяснением такого отклонения может служить зависимость $N_{at}/N_{ico}$ от процентного содержание Nb, рассмотренная  в пункте 3.1. По этой зависимости видно (рис. \ref{icocenterofico}), что ее минимум соответствует как раз процентному содержанию Nb около 45-50\%. Чем меньше это отношение, тем крепче связь между икосаэдрами. При этом нельзя говорить о том, что наблюдается зависимость только от числа икосаэдров в сплаве. Как видно из рис. \ref{icocenterofico}, максимум числа икосаэдров наблюдается при 65\% Nb в сплаве, что не соответствует минимальной критической скорости. Таким образом, можно сделать вывод, что критическая скорость зависит не от того, насколько часто икосаэдры образуются в системе и не от их числа в системе,  а  от их связи друг с другом. Чем эта связь крепче, тем ниже  критическая скорость стеклования.



