\chapter{Литературный обзор}
\section{Аморфные металлы}
\subsection{Мотивация исследований аморфных сплавов}
За прошедшие с момента опыта Клемента и др. по получению аморфного сплава $Au-Si$ \cite{Klement} 60 лет было проведено множество исследований, посвященных аморфным металлам. Интерес к металлическим стеклам обусловлен теми физическими свойствами, которыми они обладают. Во-первых, это магнитные свойства, которыми обладают, в частности, аморфные сплавы переходных и редкоземельных металлов. К этим свойствам относятся:
\begin{enumerate}
	\item Хорошее соотношение сигнал/шум в пленках из аморфных металлов (\cite{Chaudhari} на примере сплавов Gd-Co и Gd-Fe);
	\item Наличие полосовых и пузырьковых доменов при различных условиях изготовления (\cite{Chadhauri2.0} для тех же сплавов);
	\item Большое значение коэрцитивной силы при низких температурах(\cite{Rhyne} для аморфных $TbFe_2, GdFe_2$ и $YFe_2$).
\end{enumerate} 
Все это позволяет использовать металлические стекла при изготовлении диффузионных барьеров и магнитных датчиков \cite{Zolotukhin}.


Во-вторых, это высокое (по сравнению с кристаллическими сплавами) значение удельной электропроводимости, которое позволяет использовать тонкие пленки металлических стекол в интерференционных системах, пленочных пассивных и активных элементах и тд. \cite{Antonets}.


В-третьих, это механические свойства - для аморфных сплавов характерно более низкое значение упругих констант по сравнению с их кристаллическими аналогами, что можно объяснить большим межатомным расстоянием, а также высокие значения твердости и прочности (последний параметр у аморфных сплавов значительно выше, чем у сталей)\cite{Kalin}. Помимо этого, аморфные сплавы обладают 
уникальными коррозийными и другими физико-химические свойствами, что позволяет их использовать в различных практических применениях, и, соответственно, создает причины для активных исследований аморфных сплавов.
\subsection{Получение аморфных сплавов}
Производство аморфных сплавов возможно множеством различных физических и химических  методов  из газовой, жидкой и твердой фаз \cite{Glezer}, \ref{metody}. Рассмотрим основные из них.
\begin{figure}[h!]
	\begin{center}
		\includegraphics[scale= 0.6]{metody}{}{}{}
		\caption{Способы получения аморфных сплавов \cite{Sudzuki}}
		\label{metody}
	\end{center}
\end{figure}


Первый из возможных путей получения аморфных сплавов - скоростная закалка из расплавов при скоростях $10^6-10^9\text{ K/c}$. Охлаждение с такой высокой скоростью приводит к тому, что атомы не успевают перестроиться в кристаллическую структуру, вещество застывает в переходной фазе \cite{Abrosimova}. При скоростной закалке возможно использование одноваловой установки. В ней металл тонкой струей подается на массивный вал со значительно более низкой температурой. За счет разницы в температурах и объемах вала и подаваемого металла происходит скоростная закалка \ref{pech}.
\begin{figure}[h!]
	\begin{center}
		\includegraphics[scale= 0.6]{pechka}{}{}{}
		\caption{Схема получения аморфной ленты на одноваловой установке с подачей расплава а)сверху и б)снизу. 1 - индукционная плавильная печь, 2 - разливочное сопло, 3 - охлажденный валок, 4 - металлопровод, 5 - индукционный подогрев металла в металлопроводе \cite{Danilova}}
		\label{pech}
	\end{center}
\end{figure}
С помощью этого метода возможно получение тонких аморфных лент.
Второй обширный метод - это Splat-закалка. В этом методе расплавленный металл выстреливается из пушки на подложку, которая охлаждается водой. Этот метод также позволяет получать аморфные пленки со скоростями до $10^{10}\text{ K/c}$ \cite{Davies}.

Говоря о газофазных методах, необходимо упомянуть метод вакуумного напыления. Во время этого процесса  с поверхности металла, разогретого при высоком вакууме, испаряются атомы.  Далее эти атомы попадают на подолжку и прилипают к ней. Отметим, что в процессе вакуумного напыления на поверхность подложки попадают не только частицы металла, но и газа, находящегося в камере. Соответственно, структура и свойства получаемого материала будут очень сильно зависеть от степени вакуума и от того газа, который находится в вакуумной камере \cite{Sudzuki}.

Получение аморфных сплавов из кристаллической фазы происходит с помощью внесения дефектов в кристаллическую структуру вещества. Данный процесс реализуется с помощью облучения металла, воздействия ударной волны и тд. \cite{Kalin}, \cite{Sudzuki}.

\subsection{Структура аморфных металлов}
Аморфные сплавы, по определению, характеризуются отсутствием так называемого дальнего порядка -отсутствует периодичность и повторямость в отдельных частях металла \cite{Prokhorov}. При этом, в аморфных сплавах существует коррелляция между положениями атомов в пределах  двух-трех координационных сфер \cite{Glezer2002}, а значит, существует ближний порядок.  Долгое время нерешенным оставался вопрос о том, какую же структуру имеют аморфные сплавы в ближнем порядке, что за единицы структуры присутствуют в нем. Как уже говорилось во введении, изучение структуры аморфных сплавов происходит параллельно в экспериментальных и численных исследованиях. Особое внимание в них уделено изучению Парно-коррелляционной функции и многогранникам Вороного аморфной структуры. Парно-корреляционной функцией называется функция условной вероятности того, что какая-либо частица будет найдена в точке $r$ при условии, что другая частица, называемая опорной, находится в начале координат \cite{Chandler}. Вычисляется она следующим образом:
\begin{equation}
	g(r) = \frac{dn_r}{4\pi dr \rho}
\end{equation}
В этом уравнении $dn_r$ - количество частиц, которое будет обнаружено в кольце толщиной $dr$. ПКФ связывает между собой локальную и объемную плотности с помощью формулы
\begin{equation}
	\rho(r) =\rho^{bulk}g(r) 
\end{equation},
где $\rho(r)$ - локальная плотность, $\rho^{bulk}$ - объемная. 
У аморфных веществ ПКФ существенно отличается от вида ПКФ жидкости и кристаллической фазы. Для стекол наблюдается так называемое расщепление второго пика ПКФ - там, где у жидкости и твердого тела наблюдается четко очерченный второй максимум ПКФ, у аморфных веществ наблюдается два максимума и один минимум между ними. При этом подпик, отвечающий меньшему расстоянию до опорной частицы, выше второго подпика (рис. \ref{Square}) \cite{Pan},\cite{Kolotova},\cite{Liu}. 
\begin{figure}[h!]
	\begin{center}
		\includegraphics[scale = 0.5]{fig1}{}{} {}
		\caption{Характерный вид ПКФ аморфного металла. Заштрихованной областью выделена площадь под  двумя подпиками второго пика ПКФ}
		\label{Square}
	\end{center} 
\end{figure}
\thispagestyle{plain}
Помимо этого, во многих литературных источниках отмечается значительное возрастание числа икосаэдрических кластеров \ref{icosahedra_lit}а при переходе из жидкости в аморфную фазу. Икосаэдрические кластеры встречаются и в жидкой фазе, но в небольших количествах \cite{Reddy}. При переходе в аморфное состояние наблюдается резкое возрастание числа этих многогранников. В работах \cite{Sheng}, \cite{Levchenko}, \cite{Pryadilschikov} предлагается гипотеза о том, что особую роль в структуре металлических стекол играют взаимопроникающие кластеры, образующие каркас (рис. \ref{icosahedra_lit}б). Важную роль в таких системах играют атомы растворенного металла или интерметаллида, вокруг которых строятся икосаэдры - они образуют цепи, из-за чего образуются расширенные  взаимопроникающие кластеры.
\begin{figure}[h]
	\begin{minipage}[h]{0.5\linewidth}
		\center{\includegraphics[width=1\linewidth]{fig3}} a \\
	\end{minipage}
	\hfill
	\begin{minipage}[h]{0.5\linewidth}
		\center{\includegraphics[width=1\linewidth]{fig4}} б\\
	\end{minipage}
	\caption{ Пример а)  икосаэдрического кластера; б) двух взаимопроникающих кластеров (серые - атомы Nb, черные - Zr).}
	\label{icosahedra_lit}
\end{figure}
Анализировать икосаэдрическое окружение атомов в численных экспериментах удобно с помощью метода многогранников Вороного. По определению, многогранником Вороного, построенным вокруг любого центра системы (атома), называется область пространства, любая точка которого ближе к данному центру, чем к любому другому \cite{Medvedev}. Таким образом, этот метод позволяет определять количество окружающих атом соседей и их расположение в пространстве.  Обозначение многогранников происходит следующим образом: каждый тип многогранников задается набором чисел $(n_1,n_2,n_3,….n_k)$ (обычно $k=6$), где $i$ в паре  $n_{i}$ - количество вершин у грани, n - число таких граней. Так, комбинация (0,0,0,0,12,0) соответствует многограннику, образованному двенадцатью пятиугольниками. Поскольку граней с 1 или 2 вершинами быть не может, а треугольные грани встречаются очень редко, для краткости далее будем обозначать многогранники как комбинацию $(n_4,n_5,n_6)$. Важно отметить, что при анализе с помощью многогранников Вороного атому с икосаэдрическим окружением соответствует многогранник Вороного (0,12,0), в котором у центрального атома 12 ближайших соседей.
\subsection{Критерии стеклования}
Выбор методов анализа структуры обусловлен также структурными критериями стеклования, использованными в работе. Хотя IUPAC трактует аморфизацию как фазовый переход второго рода, стеклование имеет ярко выраженный кинетический характер \cite{Dandar}. По этой причине, для определения аморфной фазы необходимо использовать какие-либо косвенные критерии. К примеру, в экспериментах зачастую используют изменения коэффициентов теплоемкости \cite{tropin2015heat} и вязкости \cite{konstantinova2009kinematicheskaya}, а также изменение коэффициента Холла \cite{Kuzmenko}. В расчетах чаще всего используют следующие методы:
\begin{enumerate}
	\item Расщепление второго пика ПКФ, котором подробно рассказано в предыдущем разделе;
	\item Критерий Вендта-Абрахама;
	\item Изменение зависимости коэффициента диффузии от температуры;
	\item Изменение зависимости коэффициента вязкости от температуры;
	\item Изменение зависимости коэффициента теплоемкости от температуры.
\end{enumerate}

Также в качестве критериев стеклования возможно использовать изменение зависимости площади под вторым пиком ПКФ от температуры \cite{Kolotova} и уже упомянутое изменение зависимости числа икосаэдрических кластеров от температуры, которое используется только для металлических стекол. 

Критерий Вендта-Абрахама, предложенный в \cite{Wendt}, основан на изменении  структуры при переходе от жидкости к стеклу и соответствующему изменению ПКФ. По этому критерию,  стеклование происходит  (рис. \ref{VA_criterium_lit}) в точке изменения зависимости отношения первого минимума к первому максимуму $g_{min}/g_{max}$ ПКФ от температуры.
\begin{figure}[h!]
	\begin{center}
		\includegraphics[scale= 0.6]{fig2}{}{}{}
		\caption{Зависимость отношения первого минимума к первому максимуму ПКФ от температуры, 55\% Nb.}
		\label{VA_criterium_lit}
	\end{center}
\end{figure}
Площадь под вторым пиком ПКФ определяется по формуле
\begin{equation}
\label{eq_2}
S = \int_{r_{min}}^{r_{max}} g(r)dr
\end{equation}
где $r_{min}$  и $r_{max}$  соответствуют минимальному и максимальному расстоянию на втором пике ПКФ, значение функции в которых равно значению в минимуме второго пика ПКФ. (рис. \ref{Square}, \cite{Kolotova}).

Поскольку рассмотренные  критерии являются структурными, важной задачей является определение взаимосвязи между ними. В работах \cite{Liu}, \cite{Kolotova}, \cite{Sheng}, \cite{Pan} отмечается, что расщепление второго пика ПКФ появляется  из-за образования систем икосаэдрических кластеров. Таким образом, необходимо объяснить связь расщепления второго пика ПКФ и наличия икосаэдрического окружения атомов в стекле.
\section{Молекулярная динамика}
Перейдем к методу моделирования, использованному в работе. Метод молекулярной динамики на сегодняшний день очень широко применяется в моделировании различных химических и физических процессов. В молекулярной динамике эволюция системы описывается с помощью численного интегрирования уравненений Ньютона \cite{Hamelberg}. Нахождение координаты частицы производится с помощью алгоритма Верле. Получить основную формулу для этого алгоритма можно, разложив в ряд значения координаты на $n-1$ и $n+1$ шагах и складывая их друг с другом. Получим:
\begin{equation}
	\overrightarrow{x}(t+\Delta t) = 2\overrightarrow{x}(t) - \overrightarrow{x}(t-\Delta t)+\overrightarrow{a}(t)\Delta t^2 + O(t^4)
\end{equation}
Как можно заметить, в данном уравнении нет необходимости знать скорость частицы в заданный момент времени, но необходимо знать ускорение. Соответственно, необходимо знание сил, которые действуют на частицу. Задать их возможно с помощью потенциалов взаимодействия между частицами. Вид этого потенциала может быть различным. Исторически, первым потенциалом, который использовался в численных экспериментах, был  классический потенциал Леннарда-Джонса \cite{Lennard}:
\begin{equation}
	U(r) = 4\epsilon\left[\left( \frac{\sigma}{r}\right)^{12}-\left( \frac{\sigma}{r}\right)^{6}\right]
\end{equation}
В этом потенциале задана зависимость задана только от расстояния, и вычисление действующей силы не составляет труда. К сожалению, такой потенциал не является универсальным, и при описании взаимодействия частиц в твердом теле необходимо использовать гораздо более сложные зависимости. Отметим особо два вида потенциалов, которые используются в работе.

Первый из них - это EAM-потенциал (Embedded Atom Method Potential). Впервые он был предложен в \cite{Daw}, и на сегодняшний день активно применяется в моделировании. Этот потенциал задан в виде:
\begin{equation}
 E_{\mathrm{tot}}=\frac{1}{2} \sum_{i,j(j\neq i)} \Phi_{s_i s_j}\left(r_{i j}\right)+\sum_{i} F_{s_i}\left(\overline{\rho}_{i}\right)
\end{equation}
В этом потенциале первая компонетнта соответствует межядерному взаимодействию, а вторая - взаимодействию ядра и электронной плотности вокруг него. Этот потенциал гораздо более точно описывает взаимодействие атомов друг с другом, чем тот же Леннард-Джонс, но он все еще не учитывает угловые зависимости в расположении атомов друг относительно друга. Зато это учитывается в потенциале ADP (angle dependent potential) \cite{mishin2005phase}. Он задается функцией вида
\begin{equation}
	E_{\mathrm{tot}}= \frac{1}{2} \sum_{i, j(j \neq i)} \Phi_{s_{i} s_{j}}\left(r_{i j}\right)+\sum_{i} F_{s_{i}}\left(\overline{\rho}_{i}\right)+\frac{1}{2} \sum_{i, \alpha}\left(\mu_{i}^{\alpha}\right)^{2} +\frac{1}{2} \sum_{i, \alpha, \beta}\left(\lambda_{i}^{\alpha \beta}\right)^{2}-\frac{1}{6} \sum_{i} v_{i}^{2}
\end{equation}
Здесь присутствуют те же две компоненты, что и в EAM-потенциале, но, помимо них, здесь добавляются дипольное и квадрупольное взаимодействие, которые и отвечают за угловое взаимодействие. По этой причине потенциал вида ADP был использован в данной работе.

В конце отметим, что на сегодняшний день реализовано большое количество программных пакетов, позволяющих проводить молекулярно-динамические рассчеты. Самый популярный из них - LAMMPS\cite{Plimpton}. Это пакет для языка $C++$, который позволяет проводить моделирование для нескольких миллионов атомов \cite{Gilyaev}, при этом сами расчеты проводятся параллельно с использованием MPI. Именно он был использован в проведенном исследовании.

\section{Задачи и цели}
Целью данной работы является исследование аморфной фазы Zr-Nb, получение связи между различными структурными критериями стеклования и определение того, при каких скоростях возможно получить аморфный сплав Zr-Nb.
Для достижения данных целей необходимо выполнить следующие задачи:
\begin{enumerate}
	\item Провести молекулярно-динамическое моделирование плавления сплава Zr-Nb
	\item Провести молекулярно-динамическое моделирование охлаждения сплава Zr-Nb для различного процентного содержания Nb и различных скоростей охлаждения.
	\item Построить по этим рассчетам фазовую диаграмму в координатах <<Скорость охлаждения - процентное содержание $Nb$>>.
	\item Проанализировать по проведенным рассчетам температуры стеклования с помощью различных критериев и сравнить полученные данные друг с другом.
	\item Получить связь между расщеплением второго пика ПКФ и возрастанием числа икосаэдрических кластеров в аморфной фазе. 
\end{enumerate}



