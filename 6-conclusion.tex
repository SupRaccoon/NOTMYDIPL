\chapter*{Выводы}
\addcontentsline{toc}{chapter}{Выводы}
В работе проведено моделирование стеклования сплава Zr-Nb, исследована аморфная фаза данного сплава, ее строение и критические скорости охлаждения. Установлено, что изменение температурной зависимости числа икосаэдрических кластеров ($N_{ico}$) для сплава Zr-Nb может быть критерием стеклования. Также установлено, что расщепление второго пика перекрестной ПКФ Zr-Nb для системы Zr-Nb может являться критерием стеклования. Показана связь между расщеплением второго пика ПКФ и икосаэдрическим окружением атомов. Расщепление второго пика ПКФ объяснено через конкретные расстояния между атомами в системе взаимопроникающих кластеров.

Исследовано влияние содержания Nb и скорости охлаждения ($K$) на число многогранников Вороного. Предложено объяснение, основанное на  стерическом факторе - различии в размерах атомов Zr и Nb. Из-за большего размера Zr у него в среднем более чем 12 соседей и поэтому очень редко возможно образование кластеров с Zr в центре. Показано, что структура, состоящая  из икосаэдрических кластеров, соединенных по грани, или из взаимопроникающих кластеров, является основной для аморфного сплава Zr-Nb. Она заполняет весь объем сплава, определяет его структуру и условия формирования.  

Определен температурный диапазон стеклования по  зависимости $N_{ico}(T)$. Нижняя граница этого диапазона  совпадает с температурой появления  расщепления второго пика ПКФ. Эта температура согласуется  с температурой стеклования, определенной по критерию Вендта-Абрахама, в пределах погрешности. Также определена температура стеклования по площади под вторым пиком ПКФ, совпадающая с двумя другими критериями.  


Получена зависимость пороговой скорости $K$ от процентного содержания Nb для сплава Zr-Nb. Показано, что поведение этой зависимости совпадает с зависимостью отношения числа атомов в икосаэдрических кластерах к числу центров икосаэдров от процентного содержание Nb.