\section{Метод молекулярной динамики с волновыми пакетами}

В предыдущих разделах были рассмотрены методы классической МД и МК, а также квантовомеханический подход, реализованный в методе PIMC, в применении к моделированию систем заряженных частиц.
Методы классической МД обладают высокой скоростью вычислений и позволяют рассматривать динамические эффекты, возникающие в электронной подсистеме.  Но учет квантовых эффектов, которые неотвратимо возникают в системах с высокой плотностью, вводится приближенно, путем задания псевдопотенциала, что не всегда покрывает весь спектр возникающих явлений \cite{kelbg1964theorie}. 
Метод Монте-Карло в терминах интегралов по траекториям и DPIMC в частности, позволяет проводить "ab initio" расчеты, но обладает высокой вычислительной сложностью и не позволяет моделировать динамические явления. 
Рассмотренный в этом разделе метод  \textbf{молекулярной динамики с волновыми пакетам} (МДВП), позволяет избежать необходимости задания псевдопотенциалов электрон-ионного взаимодействия, а также улучшить точность моделирования элементарных актов соударений частиц, а при использовании антисимметризованных волновых пакетов добавляется учет обменного взаимодействия \cite{morozov2012improvement}. Метод молекулярной динамики с волновыми пакетами впервые предложен в работе \cite{klakow1994semiclassical}. Метод применен к моделированию плазмы с высокой плотностью (более $10^{23}) \mathrm{см}^{-3}$ для водородной плазмы) при низкой температуре (меньше чем $10^4 \mathrm{K}$)  \cite{graziani2012large, knaup2003wave}.

Основная идея метода МДВП состоит в рассмотрении электрона не как классической точечной частицы, а в виде гауссовского волнового пакета с варьируемой шириной:
\begin{equation}
	\label{wpmd:eq-gauss-packet}
	\phi(\vec{x},t) = \Big(\frac{3}{2\pi\gamma^2}\Big)^{3/4} \exp \Big[ -\Big( \frac{3}{4\gamma^2} - \frac{ip_\gamma}{2\hbar\gamma}(\vec{x} - \vec{r})^2 + \frac{i}{\hbar}\vec{p}(\vec{x} - \vec{r})
	\Big) \Big].
\end{equation}
В этом случае одноэлектронная волновая функция  $\phi$ зависит от 8-ми скалярных параметров: 3-х проекций классической координаты $\vec{r}$, 3-х проекций импульса $\vec{p}$, ширины пакета $\gamma$ и сопряженного импульса $p_\gamma$. Ионы рассматриваются как классические частицы в силу их относительно большой массы.

Многочастичная волновая функция системы в простейшем приближении, без учета антисимметричный волновых пакетов, имеет вид:
\begin{equation}
	\label{wpmd:eq-psi-full}
	\Psi(\{\vec{x}_k \}, t)=\prod_k \phi(\vec{x}_k, t).
\end{equation}
Эволюция системы во времени находится из вариационного принципа
\begin{equation*}
	\delta\int_{t_1}^{t_2} \bra{\Psi} i\hbar \frac{\partial}{\partial t} - \hat{H} \ket{\Psi}dt=0
\end{equation*}
Преимущество использования неантисимметризованной волновой функции заключается в том, что для нее уравнения, определяющие зависимость динамических параметров от времени, записываются в обычном для гамильтоновой динамики виде:
\begin{equation}
	\label{wpmd:eq-dynamics}
	\vec{\dot{r}}_k(t)=\frac{\partial H}{\partial \vec{p}_k}, \quad
	\vec{\dot{p}_k}(t)= - \frac{\partial H}{\partial \vec{r}_k}, \quad
	\dot{\gamma}_k(t)= \frac{\partial H}{\partial p_{\gamma_k}}, \quad
	\dot{p}_{jk}(t)=-\frac{\partial H}{\partial \gamma_k},
\end{equation}
где функция Гамильтона $H$ находится из выражения
\begin{equation}
	\label{wpmd:eq-H}
	H(\{\vec r_k\}, \{ \vec p_k \}, \{\gamma_k\}, \{p_{\gamma_k} \}) = \bra{\Psi} \hat{H} \ket{\Psi}.
\end{equation}
Данный подход позволяет полностью сохранить алгоритм МД вычислений, добавив лишь одну дополнительную степень свободы для каждого электрона, связанную с шириной его волнового пакета. В то же время за счет конечной ширины пакетов отпадает необходимость применения электрон-ионного псевдопотенциала на коротких расстояниях, т.е. оператор $\hat H$ включает только кулоновское взаимодействие:
\begin{equation}
	\label{wpmd:eq-Hop}
	\hat H = \sum_k \frac{\vec p_k^2}{2m} + \sum_{k<l} \frac{e_k e_l}{| \vec{\hat{x}}_k - \vec{\hat{x}}_l |}.
\end{equation}
Вычисление выражения (\ref{wpmd:eq-H}) c учетом (\ref{wpmd:eq-Hop}) приводит к:
\begin{equation}
	\label{wpmd:eq-real-H}
	H = \sum_k \Big(
		\frac{\vec p_k^2}{2m} + \frac{\vec p^2_{\gamma_k}}{2m}
	\Big) + \sum_k \frac{9\hbar^2}{8m\gamma^2_k}+\sum_{k,l}\frac{e_k e_l}{r_{kl}} \erf \Big(
		\frac{r_{kl}}{\sqrt{2(\gamma_k^2 + \gamma_l^2)/3}}
	\Big).
\end{equation}
Последнее слагаемое в (\ref{wpmd:eq-real-H}) имеет сходство с псевдопотенциалом (\ref{eq:md-perf}), используемым в классической МД, однако главное отличие состоит в том, что параметры, отвечающие за ширины пакетов, являются динамическими. Второе слагаемое отвечает за изменение ширины волнового пакета, и может приводить к неограниченному расплыванию пакетов \cite{morozov2012improvement}. 

Описанный подход не учитывает спинов электронов и эффектов, вносимых спин-спиновым взаимодействием. Учет подобного рода взаимодействия можно провести путем ввода псевдопотенциала, отвечающего за обменно-корреляционное взаимодействие (метод eFF), но таким образом в модель будет вносится дополнительный переменный параметр \cite{su2007excited}. Иной подход заключается в усложнении алгоритма и использовании антисимметризованой волновой функции. 

Алгоритм приобретает более сложный вид, если использовать антисимметризованную волновую функцию. Однако это позволяет до некоторой степени учесть эффекты вырождения электронного Ферми-газа. Для простоты в расчете полагается, что половина электронов имеет проекцию спина на ось $z$, равную $1/2$, а другая половина
$-1/2$, причем спин электронов не меняется с течением времени. Таким образом, система электронов разбивается на две подсистемы, каждая из которых должна быть представлена
антисимметризованной волновой функцией вида:
\begin{equation}
	\label{wpmd:eq-psi-antisim}
	\Psi(\{\vec x_i\}) = (N! \det(\vec O))^{-1} \sum_\sigma \mathrm{sign}(\sigma) \prod_i \phi_{\sigma_i}(\vec x_i),
\end{equation}
где $N$ -- число электронов в подсистеме, $\vec O$ -- матрица перекрытия, задаваемая выражением:
\begin{equation}
	\label{wpmd:eq-overlap}
	O_{ij}=\int \phi_i(\vec x)\phi_j^*(\vec x)d^3\vec x .
\end{equation}
В этом случае для описания эволюции системы выражения (\ref{wpmd:eq-dynamics}) не применимы. Вместо этого следует решать систему уравнений вида:
\begin{align}
	\label{wpmd:eq-dynamic-antisym}
	&\frac{dQ_i}{dt} = \sum_j (N^{-1})_{ij} \frac{\partial H}{\partial Q_j}, \\
	&\label{wpmd:eq-H-antisym} H = \sum _{ij} T_{ij} O_{ij} + \sum_{(i,j), (k,l)} V_{ijkl}(Y_{ki}Y_{lj} - Y_{kj}Y_{li}) + \sum_{(i,j), (k,l)}^{max} V_{ijkl}Y_{ki}Y_{lj},
\end{align}
где $Q_i$ -- набор параметров $\{ \var r, \vec p, \gamma, p_\gamma \}_i$, $\vec N$ -- норм-матрица, $\vec Y = \vec O^-1$ -- обратная матрица перекрытия, $T_{kl} = \bra{\phi_k} \hat{T_e} + \hat{T_{ei}} \ket{\phi_l}$ -- матрица, описывающая кинетическую энергию и энергию электрон-ионного  взаимодействия, $V_{klmn}=\bra{\phi_k\phi_l}\hat{V}\ket{\phi_m\phi_n}$ -- матрица электрон-электронного взаимодействия. Важно заметить что вторая сумма в (\ref{wpmd:eq-H-antisym}) берется по частицам с одинаковым спином, а третья -- по частицам с разным спином. 

Использование антисимметризации позволяет учитывать обменное взаимодействия, не прибегая к вводу дополнительных потенциалов. Но, стоит отметить, что введение антисимметризации приводит к возрастанию вычислительной сложности метода. Время выполнения программы с учетом антисимметризации при прямом алгоритме вычислении сумм (\ref{wpmd:eq-H-antisym}) растет с числом электронов как $N^4$ , в то время как без антисимметризации это время растет как $N^2$ . 

Еще одним недостатком МДВП является расплывание волновых пакетов при моделировании высокотемпературных систем. Данная проблема может быть решена путем введения ограничивающего потенциала на границах моделируемой ячейки. 