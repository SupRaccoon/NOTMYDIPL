\section{Метод Монте-Карло в терминах интегралов по траекториям}

Метод Монте-Карло в терминах интегралов по траекториям (Path integral Monte-Carlo (PIMC)) развит в рамках феймановской формулировки квантовой механики \cite{Zelener1981theor}.  
Основная идея метода заключается в представлении матрицы плотности и, следовательно, термодинамических величин в виде интегралов по траекториям. Далее будет кратко описана теоретическая часть метода и способы его применения. 

Термодинамика квантовомеханической системы полностью определена, если известна статистическая 
сумма
\begin{equation}\label{pimc:eq-stat}
	Z = \sum_n\exp(-\beta E_n),
\end{equation} 
где $\beta=1/(k_BT)$, $E_n$ --- полная энергия системы в состоянии $n$.
Для расчета статистической суммы необходимо решить уравнение Шредингера для системы многих тел, найти собственные значения оператора энергии и провести суммирование по всем состояниям дискретного и непрерывного спектра. В связи с этим формула (\ref{pimc:eq-stat}) не применима для расчета методом Монте-Карло \cite{Zamalin77}. 

Применение метода МК становится возможным, если статистическую сумму (\ref{pimc:eq-stat}) выразить через матрицу плотности 
$\rho(R,R';\beta)$ \cite{hill1960introduction}:
\begin{equation}\label{pimc:eq-stat_rho} 
	Z = \int_V \rho(R,R';\beta))dR,
\end{equation}
где матрица плотности для системы N-тел выражается через волновые функции компонентов системы $\psi_i(R)$:
\begin{equation}
	\label{pimc:eq-rho-matrix}
	\rho(R,R';\beta)=\sum_i \psi_i^*(R)e^{-\beta E_i} \psi_i(R')
\end{equation}
В этом случае термодинамические свойства системы могут быть рассчитаны как:
\begin{equation}	\label{pimc:eq-thrm-func}
	 \overline{F}  = \frac{1}{Z}\int dR dR' \bra{R}F\ket{R'}\rho(R,R';\beta).
\end{equation}

В общем случае, матрица плотности $\rho(R,R';\beta)$ неизвестна, но ее можно приближено вычислить. Исходя из группового свойства оператора матрицы плотности 
\begin{equation*}
	\hat{\rho} = \exp(-\beta\hat{H}) = \Big[\exp(- \frac{\beta}{M}\hat{H})\Big]^M,
\end{equation*}
можно записать выражение для матрицы плотности в координатном представлении
\begin{equation} \label{pimc:eq-rho-coord}
	\rho(R_0, R_M; \beta) = \int dR_1 \dots dR_{M-1} \prod_{t=1}^M \rho(R_{t-1}, R_{t}; \Delta \tau),
\end{equation}
где $\Delta \tau = \beta / M$.
Здесь под множеством точек $\{R_1, R_2, \dots, R_M\}$ подразумевается \textit{путь}, а под $\Delta \tau$ -- \textit{временной шаг} в мнимом временном пространстве.
Данное выражение позволяет, зная высокотемпературное приближение $\tilde{\rho}$ матрицы плотности, получить достаточно точное выражение для $\rho(R,R'; \beta)$, 
пригодное при меньших температурах, варьируя длину пути $M$ \cite{Zamalin77}. 
Для потенциалов, ограниченных снизу, высокотемпературное приближение $\tilde{\rho}$ может быть записано в виде \cite{feiman2013}:  
\begin{equation}
	\label{pimc:eq-rho-hitemp}
	\tilde{\rho}(R, \tau; R', \tau') = \lambda_\Delta^{-3N} \exp \Big[ -\pi \frac{|R - R'|^2}{\lambda_\Delta^2} - (\tau' - \tau)U(R) \Big],
\end{equation}
где $\lambda_\Delta^2 = 2\pi \hbar^2 (\tau' - \tau)/m$, $m$ -- масса частицы, $U(R)$ -- потенциальная энергия, входящая в гамильтониан системы. 
Следует также отметить что для получения достаточно точного значения матрицы плотности нет необходимости в использовании больших значений параметра $M$, так как
при (\ref{pimc:eq-rho-hitemp}), (\ref{pimc:eq-rho-coord}) переходит в свой классический предел уже при $M=1$. Следовательно, выбор параметра $M>1$ требуется лишь тогда, когда проявляются квантовомеханические эффекты. Приблизительную оценку параметра $M$ можно получить, используя неравенства \cite{filinov1975}:
\begin{eqnarray*}
	& M^{-1/2}\beta \lambda \Delta \Phi <<  1, \\
	& M^{-1}\beta \lambda \Delta^2 \Phi << 1,
\end{eqnarray*}
где $\lambda^2 = 2\pi \hbar^2 \beta/m$, $\Phi$ -- потенциал взаимодействия частиц. 

Далее рассмотрим способы построения марковских цепей для расчета статистической суммы в случае бозе- и ферми-систем.
Каждая частица в методе МК по траекториям представляет собой множество точек ${R_1,R_2,\dots,R_M}$, которое можно интерпретировать как замкнутую траекторию, характеризующую частицу \cite{feiman2013}.
Характерный размер такой траектории порядка тепловой длины волны $\lambda$. Совокупность траекторий, для дальнейших изложений, удобно представить в виде матрицы
\begin{equation}
	\label{pimc:eq-trmat}
	A=
	\begin{Vmatrix}
		R_1^1, &R_2^1, &\dots, &R_N^1\\
		R_1^2, &R_2^2, &\dots, &R_N^2\\
		\vdots, &\vdots, &\ddots, &\vdots \\
		R_1^M, &R_2^M, &\dots, &R_N^M\\
		R_1^{M+1}, &R_2^{M+1}, &\dots, &R_N^{M+1}\\
	\end{Vmatrix}.
\end{equation} 
Каждый столбец матрицы $A$ соответствует траектории частицы, число столбцов равно числу частиц, а последняя стока матрицы равна первой: $R_i^{M+1} \equiv R_i^1 \space (i=1\dots N)$.

Термодинамические величины выражаются через соответствующие производные от свободной энергии, которая может быть получена из статистической суммы системы (\ref{pimc:eq-stat_rho}), которая, в свою очередь, может быть представлена $3NM$-кратным интегралом от произведения высокотемпературных матриц плотности (\ref{pimc:eq-rho-coord}).  Для произвольной термодинамической функции $F(N, V, \beta, R)$ из (\ref{pimc:eq-thrm-func}):
\begin{equation}
	\label{pimc:eq-mean-term}
	\overline{F}=Z^{-1}\int_V\dots \int_V dR^1 \dots dR^M F(N,V,\beta, R^1, \dots, R^M)f,
\end{equation}
где 
\begin{equation*}
	f=\lambda_\Delta^{-3NM} \exp \Big[ -\pi \sum_{k=1}^{M-1} \frac{|R^k -R^{k+1}|^2}{\lambda^2_\Delta} -\pi \frac{|R^M - R^1|^2}{\lambda^2_\Delta} \Big] \exp\Big[-\frac{\beta}{M} \sum_{k=1}^{M} U(R^k) \Big]
\end{equation*}
Переходя от многократных интегралов к интегральным суммам, можно записать:
\begin{eqnarray}
	\label{pimc:eq-mc}
	\begin{aligned}
		& Z\approx\tilde{Z} =\sum_{A_j}u(A_j), \\
		& \overline{F} = \sum_{A_j} F(N,V, \beta, A_j)u_j, \\
		& u_j =\tilde{Z}^{-1} \exp \Big[ -\pi \sum_{k=1}^{M-1} \frac{|R^k -R^{k+1}|^2}{\lambda^2_\Delta} \Big] \times 
										 \exp\Big[-\frac{\beta}{M} \sum_{k=1}^{M} U(R^k) \Big], 							 
	\end{aligned}
\end{eqnarray},
где каждое состояние $A_j$ характеризуется матрицей траекторий (\ref{pimc:eq-trmat}), а суммы берутся по всем возможным $A_j$. Выражения (\ref{pimc:eq-mc}) можно рассчитать методом Монте-Карло \cite{zamalin1973, fillinov1973}.

Для реализации цепи Маркова вводятся единичные переходы $A_i \longrightarrow A_j$ -- изменение значения в одной из случайно выбранных позиций матрицы $A$, с учетом того что первая и последняя строка матрицы совпадает. Вероятность перехода $p_{ij}$ определяется отношением $u_j/u_i$ .
Для рассмотрения Бозе-систем, используется выражение матрицы плотности с учетом эффектов обмена:
\begin{multline}
	\label{pimc:eq-rho-bf}
	\rho(R,R';\beta) = (N!)^{-1}\sum_P (\pm 1)^{\kappa P} \int_V \dots \int_V dR^1 \dots dR^{M-1} \rho(R,0; R^1, \tau_1) \dots \\
    \rho(R^{M-1}, \tau_{M-1}; PR', \beta),
\end{multline}
где знак плюс относится к бозе, а знак минус --- к ферми системам; $\kappa P$ --- четность перестановки.
Матрица (\ref{pimc:eq-trmat}) в случае Бозе-систем принимает вид:
\begin{equation}
\label{pimc:eq-bose-trmat}
A=
	\begin{Vmatrix}
	R_1^1, &R_2^1, &\dots, &R_N^1\\
	R_1^2, &R_2^2, &\dots, &R_N^2\\
	\vdots, &\vdots, &\ddots, &\vdots \\
	R_1^M, &R_2^M, &\dots, &R_N^M\\
	PR_1^{1}, &PR_2^{1}, &\dots, &PR_N^{1}\\
	\end{Vmatrix}.
\end{equation} 
Совокупность событий $A_j$ отличается друг от друга не только значениями координат, но и перестановками в последней строке матрицы. Для реализации цепи Маркова в случае Бозе-систем вводятся два простейших перехода:
\begin{enumerate}
	\item изменение значения одной случайно выбранной координаты в матрице (\ref{pimc:eq-bose-trmat}) (при этом последняя и первая строчки в матрице считаются как одна строка).
	\item перестановка координат в последней строчке матрицы при сохранении неизменными всех значений координат в остальных (обменный шаг). Доля таких шагов $0<\alpha<1$ является произвольным параметром задачи.
\end{enumerate}
Вероятность переходов первого типа определяется соотношением $u_j/u_i$:
\begin{multline}
	\label{pimc:eq-loguij}
	\ln (u_j/u_i) = -\pi \lambda_\Delta^{-2} ( |R_l^{k-1}(i) - R_l^k(j)|^2 + | R^k_l(j) - R_l^{k+1}(i) |^2 - \\
		- | R^{k-1}_l(i) - R_l^k(i) |^2 - | R^k_l(i) - R_l^{k+1}(i) |^2 ) \\
		-\beta M^{-1} ( U(R^k(j)) - U(R^k(i))  ),
\end{multline} 
где происходило изменение координаты $k$-й точки на $l$-й траектории. Правая часть выражения состоит из двух слагаемых: первое слагаемое учитывает изменение кинетической энергии при переходе $A_i \longrightarrow A_j$, второе слагаемое учитывает изменение потенциальной энергии в переходе.
Данное выражение в случае Бозе-систем справедливо только при изменении промежуточных строк матрицы. При изменении элементов из первой строки следует учесть, что этот же элемент находится в последней строке матрицы, но, возможно, в другом столбце. В этом случае кинетическое слагаемое из (\ref{pimc:eq-loguij}) следует использовать в виде:
\begin{multline*}
	-\pi \lambda_\Delta^{-2} ( |R_t^{M}(i) - PR_l^1(j)|^2 + | R^1_l(j) - R_l^{2}(i) |^2 - \\
			- | R^{M}_t(i) - PR_t^1(i) |^2 - | R^1_l(i) - R_t^{2}(j) |^2 ),
\end{multline*}
при изменении значения $R_l^1$, если $R_l^1 = PR_t^1$.
Для вероятности переходов второго типа отношение $u_j/u_i$ выражается как:
\begin{equation}
	\label{pimc:eq-exchange-prob}
	\ln (u_j/u_i) = \pi \lambda_\Delta^{-2} \Big[ |y_l - R_l^M|^2 + |y_m - R^M_m|^2 - |y_m - R_l^M|^2 
	- |y_l - R^m_m|^2  \Big],
\end{equation}
где $l$ и $m$ -- номера частиц, концы траекторий которых переставлялись и $PR_l^1(j)=PR_m^1(i) \equiv y_m$, $PR_m^1(j)=PR_l^1(i) \equiv y_l$ \cite{zamalin1973}.

Основное оригинальное содержание описанного метода построения марковской цепи заключается в том, что цепь рассматривается не только в пространстве координат, но и в пространстве перестановок. Непосредственное суммирование вкладов от всех $N!$ перестановок практически неосуществимо, но использование описанной цепи Маркова, позволяет ограничится лишь вкладами наиболее существенных перестановок \cite{Zamalin77}.

Модификация метода, позволяющая рассматривать ферми-системы, была представлена в \cite{filinov1975}
для исследования модели двухкомпонентной плазмы, состоящей из электронов и протонов. Данная модификация метода широко применяется для моделирования неидельаной плазмы \cite{binder1996monte, kalman1998strongly, berne1998classical, filinov2003plasma, filinov2004thermodynamic}. 
Статистическая сумма двухкомпонентной системы ионов и электронов имеет вид \cite{feiman2013}: 
\begin{equation}
\label{pinc:eq-fermi-Z}
	Z(N_e, N_i, V, \beta) = \frac{1}{N_e!N_i!} \sum_\sigma \int_V dq dr \rho(q,r, \sigma; \beta),
\end{equation}
где $q \equiv \{\vec{q_1}, \dots, \vec{q_{N_i}}\}$ -- координаты ионов, а $\sigma$ и $r$ -- спин и координаты электронов соответственно.
Матрица плотности предложена в \cite{filinov1975high, filinov1976influence}:
\begin{equation}
\label{pimc:eq-rho-fermi}
\rho_s (q,[r],\beta)=C_{N_e}^s \exp(-\beta U(q, [r], \beta)) \prod_{l=1}^{M} \prod_{p=1}^{N_e} \phi^l_{pp}\det||\psi_{ab}^{M,1}||_s,
\end{equation}
где $|| \psi_{ab}^{M,1} ||_s \equiv || \exp(-\pi /  (\Delta \lambda_e^2) |(r_a - r_b) + y_a^n|^2)||_s$, \newline $U(q,[r],\beta)=U_c^p(q) + (U^e([r],\Delta\beta) + U^{ep}(q, [r], \Delta\beta))/(M+1)$, 
$y^n_a=\Delta\lambda_e\sum_{k=1}^{M}\xi_a^k$, $s$ --- число электронов с одинаковым направлением спина, $C^s_N$ --- число сочетаний из $N$ по $s$. 
Данный подход (Direct path integral Monte-Carlo) является довольно ресурсоемким, так как на каждой перестановке необходимо вычислять определитель матрицы $||\psi_{ab}^{M,1}||_s$, но он позволяет исключить из выражения (\ref{pimc:eq-rho-bf}) член с переменным знаком, который приводит к возникновению численных ошибок, связанных с суммированием большого количества знакопеременных и близких по модулю членов (проблеме знаков), что встречается в отличных от рассмотренной, реализациях метода МК (restricted path integral Monte-Carlo), и вынуждающих вносить в метод ряд допущений \cite{binder1996monte, ceperley1995path, militzer2000variational, militzer2000path}. 
Марковская цепь в общем случае строится похожим на рассмотренным для бозе-систем образом, но для увеличения скорости работы метода, рассматриваются иные подходы к построению цепей, которые позволяют быстро рассчитывать вероятности одиночных переходов, не проводя вычисления определителя $||\psi_{ab}^{M,1}||_s$ \cite{Zamalin77}. 

Отличительной чертой метода PIMC, и DPIMC в частности, является то что, этот метод не использует дополнительных физических предположений и аппроксимаций, что позволяет изучать различные физические явления "из первых принципов". Недостатком метода является возможность рассматривать только равновесные процессы в моделируемых системах. Существуют методы, имеющие схожую теоретическую основу и использующиеся в молекулярно динамическом моделировании (Path integral molecular dynamics)~\cite{marx1996ab, cao1996adiabatic}, но данные методы не позволяют моделировать неидеальную плазму, так как они не позволяют описывать ряд квантовых эффектов, возникающих в подобных системах.

 

