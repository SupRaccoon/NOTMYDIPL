\chapter{Уравнение состояния неидеальной водородной плазмы}

\section{Зависимость давления от параметра неидеальности}

Неидеальной называют плазму, в которой энергия межчастичного взаимодействия сопоставима или больше кинетической энергии движения частиц. Такую плазму можно охарактеризовать 
параметром неидеальности:
\begin{equation*}
\Gamma = \Big (\frac{4\pi}{3} \Big)^{1/3} \frac{e^2 n_e^{1/3}}{k_BT},
\end{equation*}
где $T$ -- температура, $n_e$ -- концентрация зарядов.
Неидеальная плазма может быть получена воздействием лазерных импульсов на твердотельные мишени, в частности на наноразмерные кластеры, 
электрическим разрядом в газе высокого давления, на фронте 
ударных волн.

С использованием созданного пакета МД моделирования
проводилось исследование равновесных свойств неидеальной плазмы. Моделируемая система состояла из  
электронов и однократно заряженных ионов массами $m$ и $M$, соответственно. Отношение масс равнялось $M/m = 100$, что позволяет проводить интегрирование уравнений движения
электронов и ионов с одинаковым временным шагом. При значениях $M/m >= 10^2$ данное отношение слабо влияет на динамику электронов~\cite{Morozov2005Stolknoveniya}.
Межчастичное взаимодействие описывалось псевдопотенциалами (\ref{eq:md-pcoul}) и (\ref{eq:md-perf}).

Интегрирование проводилось с использованием численной схемы Leap-Frog с временным шагом $\Delta t = 2 \times 10^{-18} $~c. На начальном этапе расчета проводилось
термостатирование системы с использованием термостата Ланжевена для получения необходимой температуры. Далее без термостата проводился расчет траектории длиной 
в $15 \times 10^6$ временных шагов, что соответствует $t \approx 30 $~пс. 

Основные равновесные характеристики системы получались путем усреднения соответствующих величин по времени. Расчет мгновенного значения давления проводился по 
формуле:
\begin{equation}\label{eq:virial}
P=\frac{2}{3V} \sum_{i = 1}^{N} \frac {m_iv_i^2}{2} - \frac{1}{3V} \sum_{i,j=1,N}^{i>j} r_{ij} \frac{\partial U(r_{ij})}{\partial r_{ij}},
\end{equation}
где $m_i$ и $v_i$ --- масса и скорость $i$-й частицы, а $r_{ij}$ --- межчастичное расстояние~\cite{allen1989computer}.
%IM: Лучше сослаться на книгу Allen & Tildesley
%IM: Заменить во всем тексте "--" на "---", если речь идет о тире

Для удобства работы с программой и интерпретации результатов использовались внутренние единицы, в которых заряд электрона и его масса равны 1, энергия 
выражается в единицах $U_0=k_BT_0$, где $T_0 = 30000 K$, давление --- в единицах $P_0=k_BT_0/V$, где $V$ --- объем моделируемой системы. Используя выражение для
$U_0$, можно получить размерности и остальных наблюдаемых величин.

Для получения уравнения состояния неидеальной плазмы были проведены расчеты давления при различных параметрах неидеальности $\Gamma$. Количество частиц в системе
определялось размером моделируемой ячейки и требованием, чтобы эффективный радиус экранирования был много меньше размеров ячейки. 
Полученные результаты представлены в работе \cite{BLLMNS-TVT14}. Благодаря использованию графических ускорителей, удалось увеличить размер моделируемой системы с $10^3$ до
$10^5$ частиц, а также провести расчеты, которые демонстрируют сходимость рассчитываемых термодинамических параметров. Рассчитанное уравнение состояния представлено 
на рис. \ref{fig:eqState}, где $P_1 = kT \Big ( \frac{kT}{e^2} \Big )^3$. Полученные в МД моделировании результаты с хорошей точностью совпадают с теоретической моделью и результатами, представленными в работе~\cite{Norman1987eq}.
\begin{figure}[h!]
	\centering
	\includegraphics[width=0.7\linewidth]{figure/H-EOS.pdf}
	\caption{\small{Зависимость давления от параметра неидеальности $\Gamma$: 1 -- результаты МД расчета, 2 -- идеальный газ, 
			3 -- уравнение состояния Дебая-Хюккеля, 4 -- теоретическая модель \cite{Norman1987eq}}}
	\label{fig:eqState}
\end{figure}

\section{Оценка области применимости классической МД}
%\section{Сравнение МД, МДВП, PIMC}

В классическом методе МД электроны и ионы представлены точечными частицами. Взаимодействие между ними описывается потенциалом Кулона на больших расстояниях и псевдопотенциалом на  маленьких. Выбор псевдопотенциала позволяет расширить диапазон применимости МД в область высоких температур и степеней ионизации. Важной характеристикой моделируемой системы является доля связанных состояний электронов и ионов, где классическое представление электронов не позволяет воспроизводить квантовые свойства возникающих состояний. Однако, область применимости классической МД с псевдопотенциалами остается не вполне определенной.

Для определения границы применимости классической МД мы рассмотрели термодинамические свойства (температура, давления) неидеальной водородной плазмы, рассчитанные методом молекулярной динамики с использованием различных псевдопотенциалов (рис.~\ref{fig:eos-potential}), и сравнили их с расчетами, проведенными методами Монте-Карло с волновыми пакетами и  PIMC~\cite{filinov2004thermodynamic}.

В данных расчетах параметры плазмы варьировались в диапазоне плотностей  $n_e = 10^{20}-10^{24}\;\mbox{см}^{-3}$ и температур $T = 10^4 - 5 \cdot 10^4$~K. Это соответствует параметрам неидеальности $\Gamma = e^2/(4 \pi \epsilon_0 k^{}_BT) (4 \pi n_e/3)^{1/3} = 0.025 - 18$ и параметру вырождения  $\Theta
=\hbar^{2} (3 \pi^2 n_e)^{2/3} / (2m_e k^{}_B T) = 0.02 - 19$. Выбранный диапазон параметров соответствует веществу в астрофизических объектах, экспериментах с сильным лазерным излучением, где водород может быть обнаружен под высоким давлением и указанными температурами. Уравнение состояния плотного водорода на данный момент также является открытым вопросом~\cite{Pierleoni-PNAS16,Desjarlais-PRB08}.

Для описания межчастичного взаимодействия в МД использовались  псевдопотенциалы типа \textbf{Erf}~(\ref{eq:md-perf}), и \textbf{Кельба}~(\ref{eq:md-pkelbg}) (рис.~\ref{fig:eos-potential}). 
\begin{table}[h]\label{tab:interaction}
	\begin{center}
		\caption{\label{tab:potentials}Межчастичные потенциалы, используемые в методе молекулярной динамики}
		\begin{tabular}{|c|c|c|c|}
			\hline
			Тип потенциала & Ион- & Электрон- & Электрон-\\
			Название модели & ионный & ионный & электронный \\
			\hline
			MD-Erf & $V^\mathrm{Coul}$, (\ref{eq:md-pcoul}) & $V^\mathrm{erf}$, (\ref{eq:md-perf}) & $V^\mathrm{erf}$, (\ref{eq:md-perf}) \\
			MD-Coul & $V^\mathrm{Coul}$, (\ref{eq:md-pcoul}) & $V^\mathrm{erf}$, (\ref{eq:md-perf}) & $V^\mathrm{Coul}$, (\ref{eq:md-pcoul}) \\
			MD-Kelbg & $V^\mathrm{Kelbg}$, (\ref{eq:md-pkelbg}) & $V^\mathrm{Kelbg}$, (\ref{eq:md-pkelbg}) & $V^\mathrm{Kelbg}$, (\ref{eq:md-pkelbg}) \\
			\hline
		\end{tabular}
	\end{center}
\end{table}

\begin{figure}	
	\centering
	\includegraphics[width=0.6\linewidth]{figure/H-npp_ur.pdf}
	\caption{\label{fig:eos-potential}Межчастичные потенциалы взаимодействия при температуре $T=10000K$. Черная линия -- кулоновский потенциал (\ref{eq:md-pcoul}),
		потенциал (\ref{eq:md-perf}) представлен красной линией для e-i взаимодействия и синей для e-e. Зеленые линии -- потенциал Кельба (\ref{eq:md-pkelbg}) для  e-e и e-i взаимодействий.}
\end{figure}

\begin{figure}[t]
	\begin{multicols}{3}
		\centering
		\includegraphics[width=1\linewidth]{figure/H-U10K.pdf} \columnbreak
		\centering
		\includegraphics[width=1\linewidth]{figure/H-U30K.pdf} \columnbreak
		\centering
		\includegraphics[width=1\linewidth]{figure/H-U50K.pdf} 
	\end{multicols}
	\caption{\label{fig:eos-ut}Полная энергия водородной плазмы для температур $T~=~10000K$~(a),
		$T~=~30000K$~(b) и $T~=~50000K$~(c). Черные кресты отображают результаты PIMC \cite{filinov2004thermodynamic}.
		Синие триугольники отображают результаты расчета классическим МД с (\ref{eq:md-pcoul}) для \textit{e-e}
		и (\ref{eq:md-perf}) для \textit{e-i}, синие круги -- потенциал (\ref{eq:md-perf}) для всех типов взаимодействия и квадраты -- потенциал Кульба (\ref{eq:md-pkelbg}) для \textit{e-e, e-i}, и \textit{i-i}.   Коричневые ромбики -- результаты метода Монте-Карло с волновыми пакетами, оранжевые точки -- метод Монте-Карло с волновыми пакетами и антисимметризацией. Черная штриховая линия отображает плотность, при которой $n\lambda^3=1$. Красная линия отображает энергию идеального газа $E = 1.5 k_B T$.
	}
\end{figure}

\begin{figure}[t]
	\centering	
	\begin{multicols}{2}
		\centering
		\includegraphics[width=0.8\linewidth]{figure/H-P_30.pdf} \columnbreak
		\centering
		\includegraphics[width=0.8\linewidth]{figure/H-P_50.pdf}
	\end{multicols}
	\caption{\label{fig:eos-pt}Давление как функция от плотности для температур $T = 30000K$ (a) и $T~=~50000K$ (b).  Черные кресты отображают результаты PIMC \cite{filinov2004thermodynamic}.
		Синие треугольники отображают результаты расчета классическим МД с (\ref{eq:md-pcoul}) для \textit{e-e}
		и (\ref{eq:md-perf}) для \textit{e-i}, синие круги -- потенциал (\ref{eq:md-perf}) для всех типов взаимодействия и квадраты -- потенциал Кульба (\ref{eq:md-pkelbg}) для \textit{e-e, e-i}, и \textit{i-i}.  Черная штриховая линия отображает плотность, при которой $n\lambda^3=1$.}
\end{figure}

\begin{figure}[ht!]
	\begin{center}
		\includegraphics[width=0.35\columnwidth]{H-GR_1.pdf}\qquad
		\includegraphics[width=0.35\columnwidth]{H-GR_3.pdf}\\
		\includegraphics[width=0.35\columnwidth]{H-GR_8.pdf}\qquad
		\includegraphics[width=0.35\columnwidth]{H-GR_9.pdf}
	\end{center}
	\caption{\label{fig:eos-rdf} Парные корреляционные функции для различных плотностей: (a) $n_e=10^{20} \mathrm{cm}^{-3}$,
		(b) $n_e=10^{21} \mathrm{cm}^{-3}$, (c) $n_e=10^{23} \mathrm{cm}^{-3}$ и (d) $n_e=3\times10^{23} \mathrm{cm}^{-3}$.
		Красная линия соответствует e-e корреляциям, синяя линия -- i-i, черная линия -- e-i. Синяя точечная линия отображает ненормализованную функцию  $r^2 g_\mathrm{ei}(r)$.}
\end{figure}

Результаты расчетов различными методами для трех температур представлены на рисунках \ref{fig:eos-ut} и \ref{fig:eos-pt}.
Из рисунков видно, что характер кривых для PIMC, МД и МКВП совпадает, однако наблюдается систематическое расхождение этих результатов, которое слабо зависит от плотности, однако увеличивается с уменьшением температуры. Возможной причиной подобного расхождения является различный механизм учета связанных состояний в рассматриваемых методах~\cite{lankin2009crossover}. Выбор псевдопотенциала в методе молекулярной динамики существенно влияет на получаемые результаты. Так, для плазмы с плотностью меньше $10^{22}~\text{см}^{-3}$ тип электрон-электронного взаимодействия оказывает слабое влияние (кривые MD-Coul, MD-Erf и MD-Kelbg), но при увеличении плотности наблюдаются значительные расхождения. Использование Кулоновского потенциала приводит к стремительному возрастанию энергии при параметре вырождения $\Theta>=1$ (кривая MD-Coul). Использование потенциала (\ref{eq:md-perf}) позволяет рассматривать системы с параметром вырождения больше 1, но при больших плотностях приводит к нефизичному слипанию электронов (кривая MD-Erf). Псевдопотенциал Кельба обеспечивает наилучшее согласование с методом PIMC, но расхождение энергии все еще остается слишком большим. Использование метода МДВП дает результаты, близкие к результатам МД с потенциалом Кельба. Также в версии без антисимметризации метод МДВП приводит к слипанию волновых пакетов при больших плотностях (кривые WPMC и AWPMC). Также при использовании AWPMD возникают численные ошибки, приводящие к слипанию всех электронов в одной точке. 

Полученные результаты позволяют выделить область применимости классической МД с различными псевдопотенциалами. Так, использование кулоновского потенциала для $e-e$ взаимодействия позволяет моделировать системы с плотностью до $10^{23}~\text{см}^{-3}$. Использование псевдопотенциалов (\ref{eq:md-perf}) и (\ref{eq:md-pkelbg}) позволяет рассматривать системы с параметром вырождения больше $\Theta > 1$. Нефизичные эффекты МД моделирования возникают при плотности больше $n_e=10^{23}~\text{см}^{-3}$ для $V^{\text{erf}}$ и $n_e=10^{23}~\text{см}^{-3}$ для потенциала Кельба. При этих плотностях среднее межчастичное взаимодействие становится равным $r_0$ и силы взаимодействия становятся равными нулю. Как следствие этого, электронная подсистема образует нефизичную структуру, которая представлена на рисунке \ref{fig:eos-rdf}d с помощью парной корреляционной функции. 

\section{Перспективы развития метода МДВП}

Улучшить описание связанных состояний в неидеальной плазме и получить, в частности, более точное уравнение состояния было бы возможно с применением метода МДВП. Однако, текущая реализация метода МДВП имеет ряд недостатков, связанных с высокими требованиями к вычислительным системам при использовании антисимметризованых волновых пакетов и с возникающими численными ошибками. Также остается ряд открытых вопросов, связанных с термализацией системы и расчетом давления. Антисимметризация позволяет учитывать обменно-корреляционную энергию, но существенно влияет на производительность метода. Альтернативой этому может быть использование теории функционала плотности для расчета обменно-корреляционной энергии. 

В этом случае полная энергия системы записывается как
\begin{equation}
E=\bra{\phi}\hat{H}\ket{\phi} = E_{kin} + E_{ie} + E_h + E_{xc},
\end{equation}
где $E_{kin}$ - кинетическая энергия, $E_{ie}$ - потенциальная энергия e-i взаимодействия, 
$E_h$ - энергия Хартри, $E_{xc}$ - обменно-корреляционная энергия. Энергия Хартри, кинетическая энергия и энергия e-i взаимодействия может быть легко вычислена. Для вычисления обменно-корреляционной энергии предлагается использовать одно из приближений, используемых в теории функционала плотности. В простейшем случае это приближения локальной плотности:
\begin{equation}
E_{xc}^{[r]} = E_x[n] + E_c[n] = \int d^3rn(r)f_{xc}(n(r)) = \int d^3rn(r)f_{x}(n(r)) + \int d^3rn(r)f_c(n(r))
\end{equation} %IM: Перенести формулу на несколько строк
где $n(r)$ --- концентрация электронов
\begin{equation}
n(r) = \sum_i \theta_{ii}(r) =  \sum_i \phi_i^*(r)\phi_i(r),
\end{equation}
$\phi(r)$ --- волновой пакет электрона (\ref{wpmd:eq-gauss-packet}).

Обменная энергия рассчитывается как $f_x(r) = -C_xn^{4/3}(\vec{r})$ где $C_x$ - некоторая константа.
Получение выражения для корреляционной энергии значительно сложнее и опирается на результаты квантовомеханического метода Монте-Карло. В работе \cite{chachiyo2017simple} представлено следующее выражение для корреляционной энергии:
\begin{equation}
f_c(r_s) = a\ln(1+b/r_s+b/r_s^2),
\end{equation}
где $r_s = (4\pi n/3)^{-1/3}$ --- размер ячейки Вигнера-Зельца, $a = \frac{\ln2 - 1}{ 2\pi^2}$, b = 20.4562557.

Расчет $E_{xc}$ в данном подходе требует введение пространственной сетки для интегрирования по пространству, что может привести к еще большему повышению требований к вычислительным системам, но методы, использующие пространственные сетки с высокой эффективностью, могут быть реализованы на GPU, тем самым позволив значительно ускорить расчеты методом МДВП.
%IM: Далее надо написать, в каком состоянии находится эта работа сейчас и сделать какой-то вывод (стоит ли ее продолжать)
