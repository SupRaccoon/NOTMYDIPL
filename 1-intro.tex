\chapter*{Введение}
\addcontentsline{toc}{chapter}{Введение}

В зависимости от скорости охлаждения расплава металлов возможно получить три вида фаз - нанокристаллическую, в которой большая часть вещества находится в кристаллической фазе, переходную, в которой вещество в целом аморфно, но уже появляются зародыши кристаллической фазы, и аморфную {\cite{Zolotukhin}}. Металлы в аморфной фазе на сегодняшний день широко используются на практике. Они применяются в различных областях промышленности - от оборонной (производстве защитных бронированных  ограждений) до производства бытовой техники. Причинами такого активного использования металлических стекол являются особые физические свойства этих соединений, обусловленные некристаллической структурой {\cite{Zolotukhin}}. Изучение аморфных металлов ведется уже более полувека с помощью теоретических \cite{Egami}, \cite{Gaskell}, экспериментальных \cite{Waseda} и численных подходов (к примеру, \cite{Pisarev} -  \cite{Kolotova}). Особое внимание уделено изучению структуры аморфных соединение и сравнению результатов, получаемых с помощью различных методов \cite{Reddy}, \cite{Sheng}. Чаще  всего используют парно-корреляционную функцию (ПКФ), которую в экспериментальных работах можно получить с помощью рентгеноструктурного анализа \cite{Waseda}, \cite{Liu}, \cite{Hoare} и многогранники Вороного (к примеру,\cite{Wei} -  \cite{Evteev}).

В данной работе исследуется сплав  Zr-Nb. Для него хорошо исследована кристаллическая фаза, которая уже активно используется  для производства корпусов ТВЭЛов и конструкционных изделий ТВС \cite{Gordeev} - \cite{Derevyako},  а также в имплантологии. Широкое использование данного сплава обусловлено хорошими корозийными параметрами и деформационной стойкостью, которыми он обладает. Тем не менее, аморфная фаза данного сплава, свойства которой определяются некристаллической структурой, в настоящее время мало изучена. По этой причине важно исследовать условия получения аморфного сплава Zr-Nb и особенности его строения, определяющие его свойства.  Возможность проведения молекулярно-динамического моделирования стеклования расплава Zr-Nb появилась только в 2017 году после разработки потенциала  \cite{Smirnova}, поскольку существовавший ранее потенциал взаимодействия \cite{lin2013n} не описывает удовлетворительно некоторые параметры, к примеру температуры плавления и коэффициенты линейного и объемного расширения.
%Помимо ограниченной в пространстве кластерной наноплазмы, особый интерес представляет пространственно однородная, водородная плазма. Методы классической МД с %различными псевдопотенциалами и МДВП были применены для моделирования водородной плазмы в диапазоне плотностей  $n_e = 10^{20}-10^{24}\;\mbox{см}^{-3}$ и температур %$T = 10^4 - 5 \cdot 10^4$~K. Результаты моделирования сравнивались с результатами полученными методом Path Integral Monte-Carlo (PIMC). 