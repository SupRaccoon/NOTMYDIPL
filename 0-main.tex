\documentclass[master,14pt,subf,href,colorlinks=true]{disser}

\usepackage[utf8]{inputenc}
\usepackage{graphicx}
\usepackage{mathtools}
\usepackage{amsmath}
\usepackage{physics}
\usepackage{multicol}

\usepackage[a4paper, mag=1000, includefoot, left=3cm, right=2cm, top=2cm, bottom=2cm, headsep=1cm, footskip=1cm]{geometry}
\usepackage[T2A]{fontenc}
\usepackage[english,russian]{babel}
\usepackage{epstopdf}
%\usepackage[options]{natbib}


%\usepackage[%
%backend=biber,% движок
%bibencoding=utf8,% кодировка bib файла
%sorting=none,% настройка сортировки списка литературы
%style=gost-numeric,% стиль цитирования и библиографии (по ГОСТ)
%language=autobib,% получение языка из babel/polyglossia, default: autobib % если ставить autocite или auto, то цитаты в тексте с указанием страницы, получат указание страницы на языке оригинала
%autolang=other,% многоязычная библиография
%clearlang=true,% внутренний сброс поля language, если он совпадает с языком из babel/polyglossia
%defernumbers=true,% нумерация проставляется после двух компиляций, зато позволяет выцеплять библиографию по ключевым словам и нумеровать не из большего списка
%sortcites=true,% сортировать номера затекстовых ссылок при цитировании (если в квадратных скобках несколько ссылок, то отображаться будут отсортированно, а не абы как)
%doi=false,% Показывать или нет ссылки на DOI
%isbn=false,% Показывать или нет ISBN, ISSN, ISRN
%]{biblatex}

%\addbibresource{lavrinenko.bib}
%\addbibresource{MBT18Lavrinenko.bib}
%\addbibresource{Elbrus17.bib}
\hyphenpenalty=1000

% Номера страниц сверху и по центру
%\def\headfont{\small}
%\pagestyle{headcenter}
%\pagestyle{plain} % нумерация вкл

% Точка с запятой в качестве разделителя между номерами цитирований
%\setcitestyle{semicolon}

% Использовать полужирное начертание для векторов
\let\vec=\mathbf

% Включать подсекции в оглавление
\setcounter{tocdepth}{2}

\graphicspath{{figure/}}

\pagestyle{headcenter}
%\pagestyle{plain} 
\begin{document}
\begin{titlepage}
	\newpage
	\begin{center}
		\thispagestyle{empty}
		
		\vspace{2cm}
		{Министерство образования и науки Российской Федерации\\} 
		\vspace{0.5cm}
		{Федеральное государственное автономное образовательное учреждение \\ высшего профессионального образования \\ «Московский физико-технический институт \\ 
			(государственный университет)»\\}
		\vspace{1cm}
		{Физтех-школа Электроники, Фотоники и Молекулярной физики\\}
		{Кафедра физики высокотемпературных процессов\\}
		
		\vspace{2cm}
		\bfseries
		{Молекулярно-динамическое моделирование стеклования переохлажденного расплава Zr-Nb.\\}
		\mdseries
		\vspace{0.5cm}
		{Выпускная квалификационная работа\\(бакалаврская работа)}
		\vspace{1.5cm}
		\begin{flushleft}
			{\noindent\hspace{7cm} Выполнил:\\}
			{\noindent\hspace{7cm} студент группы 642б\\
			 \noindent\hspace{7cm} Клявинек Сергей Сергеевич\\}
			\noindent\hspace{7cm} \hrulefill\\
			{\noindent\hspace{7cm} Научный руководитель:\\}
			{\noindent\hspace{7cm} к.ф.-м.н. Колотова Лада Николаевна\\}
			\noindent\hspace{7cm} \hrulefill
		\end{flushleft}
		\vspace{\fill}
		\begin{center}
			{Москва 2020\\}
		\end{center}
	\end{center}
\end{titlepage}
\color{black}{\tableofcontents}
%\pagestyle{empty}
\clearpage
\color{black}


%Сокращения
\chapter*{Обозначения и сокращения}
ПКФ - парно-коррелляционная функция

ТВЭЛ - Тепловыделяющий элемент

ТВС - Тепловыделяющая сборка

$g_{min}$ - первый минимум ПКФ

$g_{max}$- первый максимум ПКФ

МД - молекулярная динамика

EAM - Embedded Atom Method

ADP - Angle Dependent Potential

$N_{at}$ - количество атомов в икосаэдрической структуре

$N_{ico}$ - количество атомов с икосаэдрическим окружением в икосаэдрической структуре

$\text{ПКФ}_{ico}$ - ПКФ атомов икосаэдрической структуры

$\text{П}_2$ - второй пик ПКФ

$\text{ПКФ}_{ooico}$ - ПКФ атомов вне икосаэдрической структуры

$\text{П}_2^2$ - второй подпик  второго пика ПКФ

$\text{П}_2^1$ - первый подпик  второго пика ПКФ

$\text{П}_1$ - первый пик ПКФ

$T_g$ - температура стеклования

$S$ - площадь под вторым пиком ПКФ

$S_{max}$ - максимальное значение площади под вторым пиком ПКФ

$K$ - скорость охлаждения





%\chapter*{Введение}
\chapter*{Введение}
\addcontentsline{toc}{chapter}{Введение}

В зависимости от скорости охлаждения расплава металлов возможно получить три вида фаз - нанокристаллическую, в которой большая часть вещества находится в кристаллической фазе, переходную, в которой вещество в целом аморфно, но уже появляются зародыши кристаллической фазы, и аморфную {\cite{Zolotukhin}}. Металлы в аморфной фазе на сегодняшний день широко используются на практике. Они применяются в различных областях промышленности - от оборонной (производстве защитных бронированных  ограждений) до производства бытовой техники. Причинами такого активного использования металлических стекол являются особые физические свойства этих соединений, обусловленные некристаллической структурой {\cite{Zolotukhin}}. Изучение аморфных металлов ведется уже более полувека с помощью теоретических \cite{Egami}, \cite{Gaskell}, экспериментальных \cite{Waseda} и численных подходов (к примеру, \cite{Pisarev} -  \cite{Kolotova}). Особое внимание уделено изучению структуры аморфных соединение и сравнению результатов, получаемых с помощью различных методов \cite{Reddy}, \cite{Sheng}. Чаще  всего используют парно-корреляционную функцию (ПКФ), которую в экспериментальных работах можно получить с помощью рентгеноструктурного анализа \cite{Waseda}, \cite{Liu}, \cite{Hoare} и многогранники Вороного (к примеру,\cite{Wei} -  \cite{Evteev}).

В данной работе исследуется сплав  Zr-Nb. Для него хорошо исследована кристаллическая фаза, которая уже активно используется  для производства корпусов ТВЭЛов и конструкционных изделий ТВС \cite{Gordeev} - \cite{Derevyako},  а также в имплантологии. Широкое использование данного сплава обусловлено хорошими корозийными параметрами и деформационной стойкостью, которыми он обладает. Тем не менее, аморфная фаза данного сплава, свойства которой определяются некристаллической структурой, в настоящее время мало изучена. По этой причине важно исследовать условия получения аморфного сплава Zr-Nb и особенности его строения, определяющие его свойства.  Возможность проведения молекулярно-динамического моделирования стеклования расплава Zr-Nb появилась только в 2017 году после разработки потенциала  \cite{Smirnova}, поскольку существовавший ранее потенциал взаимодействия \cite{lin2013n} не описывает удовлетворительно некоторые параметры, к примеру температуры плавления и коэффициенты линейного и объемного расширения.
%Помимо ограниченной в пространстве кластерной наноплазмы, особый интерес представляет пространственно однородная, водородная плазма. Методы классической МД с %различными псевдопотенциалами и МДВП были применены для моделирования водородной плазмы в диапазоне плотностей  $n_e = 10^{20}-10^{24}\;\mbox{см}^{-3}$ и температур %$T = 10^4 - 5 \cdot 10^4$~K. Результаты моделирования сравнивались с результатами полученными методом Path Integral Monte-Carlo (PIMC). 

%литобзор
\chapter{Литературный обзор}
\section{Аморфные металлы}
\subsection{Мотивация исследований аморфных сплавов}
За прошедшие с момента опыта Клемента и др. по получению аморфного сплава $Au-Si$ \cite{Klement} 60 лет было проведено множество исследований, посвященных аморфным металлам. Интерес к металлическим стеклам обусловлен теми физическими свойствами, которыми они обладают. Во-первых, это магнитные свойства, которыми обладают, в частности, аморфные сплавы переходных и редкоземельных металлов. К этим свойствам относятся:
\begin{enumerate}
	\item Хорошее соотношение сигнал/шум в пленках из аморфных металлов (\cite{Chaudhari} на примере сплавов Gd-Co и Gd-Fe);
	\item Наличие полосовых и пузырьковых доменов при различных условиях изготовления (\cite{Chadhauri2.0} для тех же сплавов);
	\item Большое значение коэрцитивной силы при низких температурах(\cite{Rhyne} для аморфных $TbFe_2, GdFe_2$ и $YFe_2$).
\end{enumerate} 
Все это позволяет использовать металлические стекла при изготовлении диффузионных барьеров и магнитных датчиков \cite{Zolotukhin}.


Во-вторых, это высокое (по сравнению с кристаллическими сплавами) значение удельной электропроводимости, которое позволяет использовать тонкие пленки металлических стекол в интерференционных системах, пленочных пассивных и активных элементах и тд. \cite{Antonets}.


В-третьих, это механические свойства - для аморфных сплавов характерно более низкое значение упругих констант по сравнению с их кристаллическими аналогами, что можно объяснить большим межатомным расстоянием, а также высокие значения твердости и прочности (последний параметр у аморфных сплавов значительно выше, чем у сталей)\cite{Kalin}. Помимо этого, аморфные сплавы обладают 
уникальными коррозийными и другими физико-химические свойствами, что позволяет их использовать в различных практических применениях, и, соответственно, создает причины для активных исследований аморфных сплавов.
\subsection{Получение аморфных сплавов}
Производство аморфных сплавов возможно множеством различных физических и химических  методов  из газовой, жидкой и твердой фаз \cite{Glezer}, \ref{metody}. Рассмотрим основные из них.
\begin{figure}[h!]
	\begin{center}
		\includegraphics[scale= 0.6]{metody}{}{}{}
		\caption{Способы получения аморфных сплавов \cite{Sudzuki}}
		\label{metody}
	\end{center}
\end{figure}


Первый из возможных путей получения аморфных сплавов - скоростная закалка из расплавов при скоростях $10^6-10^9\text{ K/c}$. Охлаждение с такой высокой скоростью приводит к тому, что атомы не успевают перестроиться в кристаллическую структуру, вещество застывает в переходной фазе \cite{Abrosimova}. При скоростной закалке возможно использование одноваловой установки. В ней металл тонкой струей подается на массивный вал со значительно более низкой температурой. За счет разницы в температурах и объемах вала и подаваемого металла происходит скоростная закалка \ref{pech}.
\begin{figure}[h!]
	\begin{center}
		\includegraphics[scale= 0.6]{pechka}{}{}{}
		\caption{Схема получения аморфной ленты на одноваловой установке с подачей расплава а)сверху и б)снизу. 1 - индукционная плавильная печь, 2 - разливочное сопло, 3 - охлажденный валок, 4 - металлопровод, 5 - индукционный подогрев металла в металлопроводе \cite{Danilova}}
		\label{pech}
	\end{center}
\end{figure}
С помощью этого метода возможно получение тонких аморфных лент.
Второй обширный метод - это Splat-закалка. В этом методе расплавленный металл выстреливается из пушки на подложку, которая охлаждается водой. Этот метод также позволяет получать аморфные пленки со скоростями до $10^{10}\text{ K/c}$ \cite{Davies}.

Говоря о газофазных методах, необходимо упомянуть метод вакуумного напыления. Во время этого процесса  с поверхности металла, разогретого при высоком вакууме, испаряются атомы.  Далее эти атомы попадают на подолжку и прилипают к ней. Отметим, что в процессе вакуумного напыления на поверхность подложки попадают не только частицы металла, но и газа, находящегося в камере. Соответственно, структура и свойства получаемого материала будут очень сильно зависеть от степени вакуума и от того газа, который находится в вакуумной камере \cite{Sudzuki}.

Получение аморфных сплавов из кристаллической фазы происходит с помощью внесения дефектов в кристаллическую структуру вещества. Данный процесс реализуется с помощью облучения металла, воздействия ударной волны и тд. \cite{Kalin}, \cite{Sudzuki}.

\subsection{Структура аморфных металлов}
Аморфные сплавы, по определению, характеризуются отсутствием так называемого дальнего порядка -отсутствует периодичность и повторямость в отдельных частях металла \cite{Prokhorov}. При этом, в аморфных сплавах существует коррелляция между положениями атомов в пределах  двух-трех координационных сфер \cite{Glezer2002}, а значит, существует ближний порядок.  Долгое время нерешенным оставался вопрос о том, какую же структуру имеют аморфные сплавы в ближнем порядке, что за единицы структуры присутствуют в нем. Как уже говорилось во введении, изучение структуры аморфных сплавов происходит параллельно в экспериментальных и численных исследованиях. Особое внимание в них уделено изучению Парно-коррелляционной функции и многогранникам Вороного аморфной структуры. Парно-корреляционной функцией называется функция условной вероятности того, что какая-либо частица будет найдена в точке $r$ при условии, что другая частица, называемая опорной, находится в начале координат \cite{Chandler}. Вычисляется она следующим образом:
\begin{equation}
	g(r) = \frac{dn_r}{4\pi dr \rho}
\end{equation}
В этом уравнении $dn_r$ - количество частиц, которое будет обнаружено в кольце толщиной $dr$. ПКФ связывает между собой локальную и объемную плотности с помощью формулы
\begin{equation}
	\rho(r) =\rho^{bulk}g(r) 
\end{equation},
где $\rho(r)$ - локальная плотность, $\rho^{bulk}$ - объемная. 
У аморфных веществ ПКФ существенно отличается от вида ПКФ жидкости и кристаллической фазы. Для стекол наблюдается так называемое расщепление второго пика ПКФ - там, где у жидкости и твердого тела наблюдается четко очерченный второй максимум ПКФ, у аморфных веществ наблюдается два максимума и один минимум между ними. При этом подпик, отвечающий меньшему расстоянию до опорной частицы, выше второго подпика (рис. \ref{Square}) \cite{Pan},\cite{Kolotova},\cite{Liu}. 
\begin{figure}[h!]
	\begin{center}
		\includegraphics[scale = 0.5]{fig1}{}{} {}
		\caption{Характерный вид ПКФ аморфного металла. Заштрихованной областью выделена площадь под  двумя подпиками второго пика ПКФ}
		\label{Square}
	\end{center} 
\end{figure}
\thispagestyle{plain}
Помимо этого, во многих литературных источниках отмечается значительное возрастание числа икосаэдрических кластеров \ref{icosahedra_lit}а при переходе из жидкости в аморфную фазу. Икосаэдрические кластеры встречаются и в жидкой фазе, но в небольших количествах \cite{Reddy}. При переходе в аморфное состояние наблюдается резкое возрастание числа этих многогранников. В работах \cite{Sheng}, \cite{Levchenko}, \cite{Pryadilschikov} предлагается гипотеза о том, что особую роль в структуре металлических стекол играют взаимопроникающие кластеры, образующие каркас (рис. \ref{icosahedra_lit}б). Важную роль в таких системах играют атомы растворенного металла или интерметаллида, вокруг которых строятся икосаэдры - они образуют цепи, из-за чего образуются расширенные  взаимопроникающие кластеры.
\begin{figure}[h]
	\begin{minipage}[h]{0.5\linewidth}
		\center{\includegraphics[width=1\linewidth]{fig3}} a \\
	\end{minipage}
	\hfill
	\begin{minipage}[h]{0.5\linewidth}
		\center{\includegraphics[width=1\linewidth]{fig4}} б\\
	\end{minipage}
	\caption{ Пример а)  икосаэдрического кластера; б) двух взаимопроникающих кластеров (серые - атомы Nb, черные - Zr).}
	\label{icosahedra_lit}
\end{figure}
Анализировать икосаэдрическое окружение атомов в численных экспериментах удобно с помощью метода многогранников Вороного. По определению, многогранником Вороного, построенным вокруг любого центра системы (атома), называется область пространства, любая точка которого ближе к данному центру, чем к любому другому \cite{Medvedev}. Таким образом, этот метод позволяет определять количество окружающих атом соседей и их расположение в пространстве.  Обозначение многогранников происходит следующим образом: каждый тип многогранников задается набором чисел $(n_1,n_2,n_3,….n_k)$ (обычно $k=6$), где $i$ в паре  $n_{i}$ - количество вершин у грани, n - число таких граней. Так, комбинация (0,0,0,0,12,0) соответствует многограннику, образованному двенадцатью пятиугольниками. Поскольку граней с 1 или 2 вершинами быть не может, а треугольные грани встречаются очень редко, для краткости далее будем обозначать многогранники как комбинацию $(n_4,n_5,n_6)$. Важно отметить, что при анализе с помощью многогранников Вороного атому с икосаэдрическим окружением соответствует многогранник Вороного (0,12,0), в котором у центрального атома 12 ближайших соседей.
\subsection{Критерии стеклования}
Выбор методов анализа структуры обусловлен также структурными критериями стеклования, использованными в работе. Хотя IUPAC трактует аморфизацию как фазовый переход второго рода, стеклование имеет ярко выраженный кинетический характер \cite{Dandar}. По этой причине, для определения аморфной фазы необходимо использовать какие-либо косвенные критерии. К примеру, в экспериментах зачастую используют изменения коэффициентов теплоемкости \cite{tropin2015heat} и вязкости \cite{konstantinova2009kinematicheskaya}, а также изменение коэффициента Холла \cite{Kuzmenko}. В расчетах чаще всего используют следующие методы:
\begin{enumerate}
	\item Расщепление второго пика ПКФ, котором подробно рассказано в предыдущем разделе;
	\item Критерий Вендта-Абрахама;
	\item Изменение зависимости коэффициента диффузии от температуры;
	\item Изменение зависимости коэффициента вязкости от температуры;
	\item Изменение зависимости коэффициента теплоемкости от температуры.
\end{enumerate}

Также в качестве критериев стеклования возможно использовать изменение зависимости площади под вторым пиком ПКФ от температуры \cite{Kolotova} и уже упомянутое изменение зависимости числа икосаэдрических кластеров от температуры, которое используется только для металлических стекол. 

Критерий Вендта-Абрахама, предложенный в \cite{Wendt}, основан на изменении  структуры при переходе от жидкости к стеклу и соответствующему изменению ПКФ. По этому критерию,  стеклование происходит  (рис. \ref{VA_criterium_lit}) в точке изменения зависимости отношения первого минимума к первому максимуму $g_{min}/g_{max}$ ПКФ от температуры.
\begin{figure}[h!]
	\begin{center}
		\includegraphics[scale= 0.6]{fig2}{}{}{}
		\caption{Зависимость отношения первого минимума к первому максимуму ПКФ от температуры, 55\% Nb.}
		\label{VA_criterium_lit}
	\end{center}
\end{figure}
Площадь под вторым пиком ПКФ определяется по формуле
\begin{equation}
\label{eq_2}
S = \int_{r_{min}}^{r_{max}} g(r)dr
\end{equation}
где $r_{min}$  и $r_{max}$  соответствуют минимальному и максимальному расстоянию на втором пике ПКФ, значение функции в которых равно значению в минимуме второго пика ПКФ. (рис. \ref{Square}, \cite{Kolotova}).

Поскольку рассмотренные  критерии являются структурными, важной задачей является определение взаимосвязи между ними. В работах \cite{Liu}, \cite{Kolotova}, \cite{Sheng}, \cite{Pan} отмечается, что расщепление второго пика ПКФ появляется  из-за образования систем икосаэдрических кластеров. Таким образом, необходимо объяснить связь расщепления второго пика ПКФ и наличия икосаэдрического окружения атомов в стекле.
\section{Молекулярная динамика}
Перейдем к методу моделирования, использованному в работе. Метод молекулярной динамики на сегодняшний день очень широко применяется в моделировании различных химических и физических процессов. В молекулярной динамике эволюция системы описывается с помощью численного интегрирования уравненений Ньютона \cite{Hamelberg}. Нахождение координаты частицы производится с помощью алгоритма Верле. Получить основную формулу для этого алгоритма можно, разложив в ряд значения координаты на $n-1$ и $n+1$ шагах и складывая их друг с другом. Получим:
\begin{equation}
	\overrightarrow{x}(t+\Delta t) = 2\overrightarrow{x}(t) - \overrightarrow{x}(t-\Delta t)+\overrightarrow{a}(t)\Delta t^2 + O(t^4)
\end{equation}
Как можно заметить, в данном уравнении нет необходимости знать скорость частицы в заданный момент времени, но необходимо знать ускорение. Соответственно, необходимо знание сил, которые действуют на частицу. Задать их возможно с помощью потенциалов взаимодействия между частицами. Вид этого потенциала может быть различным. Исторически, первым потенциалом, который использовался в численных экспериментах, был  классический потенциал Леннарда-Джонса \cite{Lennard}:
\begin{equation}
	U(r) = 4\epsilon\left[\left( \frac{\sigma}{r}\right)^{12}-\left( \frac{\sigma}{r}\right)^{6}\right]
\end{equation}
В этом потенциале задана зависимость задана только от расстояния, и вычисление действующей силы не составляет труда. К сожалению, такой потенциал не является универсальным, и при описании взаимодействия частиц в твердом теле необходимо использовать гораздо более сложные зависимости. Отметим особо два вида потенциалов, которые используются в работе.

Первый из них - это EAM-потенциал (Embedded Atom Method Potential). Впервые он был предложен в \cite{Daw}, и на сегодняшний день активно применяется в моделировании. Этот потенциал задан в виде:
\begin{equation}
 E_{\mathrm{tot}}=\frac{1}{2} \sum_{i,j(j\neq i)} \Phi_{s_i s_j}\left(r_{i j}\right)+\sum_{i} F_{s_i}\left(\overline{\rho}_{i}\right)
\end{equation}
В этом потенциале первая компонетнта соответствует межядерному взаимодействию, а вторая - взаимодействию ядра и электронной плотности вокруг него. Этот потенциал гораздо более точно описывает взаимодействие атомов друг с другом, чем тот же Леннард-Джонс, но он все еще не учитывает угловые зависимости в расположении атомов друг относительно друга. Зато это учитывается в потенциале ADP (angle dependent potential) \cite{mishin2005phase}. Он задается функцией вида
\begin{equation}
	E_{\mathrm{tot}}= \frac{1}{2} \sum_{i, j(j \neq i)} \Phi_{s_{i} s_{j}}\left(r_{i j}\right)+\sum_{i} F_{s_{i}}\left(\overline{\rho}_{i}\right)+\frac{1}{2} \sum_{i, \alpha}\left(\mu_{i}^{\alpha}\right)^{2} +\frac{1}{2} \sum_{i, \alpha, \beta}\left(\lambda_{i}^{\alpha \beta}\right)^{2}-\frac{1}{6} \sum_{i} v_{i}^{2}
\end{equation}
Здесь присутствуют те же две компоненты, что и в EAM-потенциале, но, помимо них, здесь добавляются дипольное и квадрупольное взаимодействие, которые и отвечают за угловое взаимодействие. По этой причине потенциал вида ADP был использован в данной работе.

В конце отметим, что на сегодняшний день реализовано большое количество программных пакетов, позволяющих проводить молекулярно-динамические рассчеты. Самый популярный из них - LAMMPS\cite{Plimpton}. Это пакет для языка $C++$, который позволяет проводить моделирование для нескольких миллионов атомов \cite{Gilyaev}, при этом сами расчеты проводятся параллельно с использованием MPI. Именно он был использован в проведенном исследовании.

\section{Задачи и цели}
Целью данной работы является исследование аморфной фазы Zr-Nb, получение связи между различными структурными критериями стеклования и определение того, при каких скоростях возможно получить аморфный сплав Zr-Nb.
Для достижения данных целей необходимо выполнить следующие задачи:
\begin{enumerate}
	\item Провести молекулярно-динамическое моделирование плавления сплава Zr-Nb
	\item Провести молекулярно-динамическое моделирование охлаждения сплава Zr-Nb для различного процентного содержания Nb и различных скоростей охлаждения.
	\item Построить по этим рассчетам фазовую диаграмму в координатах <<Скорость охлаждения - процентное содержание $Nb$>>.
	\item Проанализировать по проведенным рассчетам температуры стеклования с помощью различных критериев и сравнить полученные данные друг с другом.
	\item Получить связь между расщеплением второго пика ПКФ и возрастанием числа икосаэдрических кластеров в аморфной фазе. 
\end{enumerate}




%\chapter{Методы атомистического моделирования}
\chapter{Экспериментальная часть}
\section{Параметры расчета}
Моделирование в данной  работе проводилось методом молекулярной динамики (МД). В качестве потенциала взаимодействия был взят потенциал погруженного атома с угловой зависимостью (ADP-потенциал \cite{mishin2005phase}) с параметризацией \cite{Smirnova}. Все расчеты были проведены  для кубической расчетной ячейки размером $21,5*21,5*21,5 \text{ нм}^3$  в NVE - ансамбле, для диапазона процентного содержания  Nb от 0 до 100 процентов и скорости охлаждения от 0,5 до 11 K/пс.  Для устранения поверхностных эффектов применялись периодические граничные условия. В расчетной ячейке поддерживалось постоянное давление посредством оставленного в ней пустого пространства, сплав перед началом плавления занимал объем $21,5*14,3*21,5 \text{ нм}^3$, температура структуры составляла $300 $ K, нагрев проводился до $2500$ K. Шаг интегрирования составлял $5 \cdot10^{-3}\text{ пс}$. Все расчеты проведены с использованием пакета LAMMPS \cite{Plimpton}.
\section{Анализ структуры}
\subsection{Анализ икосаэдрического окружения}
Анализ многогранников Вороного проводился с помощью программы OVITO \cite{Stukowski}. При анализе многогранников (0,12,0) отмечены следующие особенности:
\begin{enumerate}
	\item Такие многогранники  наблюдаются в аморфной фазе Zr-Nb, хотя и в достаточно небольшом количестве - максимальное процентное содержание порядка 6,8\% от общего числа многогранников.
	\item Их число резко возрастает при переходе «жидкость - стекло».
	\item Многогранники вида (0,12,0) образует практически полностью только Nb. Максимальное значение числа многогранников - порядка 14\% от числа атомов Nb. 
	\item Zr образует такие многогранники гораздо реже - в его случае максимальное число икосаэдрических кластеров не превышает 2\% процентов от общего числа атомов Zr.
\end{enumerate}

Для объяснения того, что икосаэдрические кластеры почти полностью образуются только вокруг Nb, был использован стерический фактор, которым объяснялось аналогичное поведение компонентов аморфного сплава в работах  \cite{Sheng},\cite{Fukunaga}, \cite{Pryadilschikov} и тд.  Важное влияние оказывает относительный размер атомов, который в случае металлов играет роль длины связи. Дело в том, что чем больше разница в размерах атомов, содержащихся в сплаве, тем больше соседей возможно разместить вокруг атома. Соответственно, чем больше один из атомов по сравнению со вторым, тем больше будет соседей у него, и, соответственно, тем меньше соседей  у атома с меньшим радиусом. Из работы \cite{Tretyakov}  радиус Zr равен 160 пм, Nb - 145 пм. Соответственно,  радиус атома Zr больше, и вокруг него расположено более чем 12 атомов. К аналогичным выводам пришли для другого аморфного бинарного сплава в работе \cite{Bondarev}.

Для проверки данной гипотезы было проанализировано распределение атомов по числу соседей. Оказалось, что у Nb в среднем порядка 12-13 соседей (более трети атомов с 12 соседями), а в случае Zr менее 8 \% атомов имеет 12 соседей - больше всего атомов с 14-15 соседями (рис. \ref{chislo_sosedey}). 
\begin{figure}[h!]
	\begin{center}
		\includegraphics[scale= 0.5]{fig5}{}{}{}
		\caption{Распределение атомов по числу соседей  для Zr и Nb (40\% Nb).}
		\label{chislo_sosedey}
	\end{center}
\end{figure}

Также для сравнения были рассмотрены аморфные сплавы Ni-Nb и Ni-Zr. Моделирование было проведено для тех же условий с помощью EAM-потенциалов  \cite{Zhang}, \cite{Mendelev}. Радиус Ni 124 пм \cite{Tretyakov}, и в этих структурах он являлся меньшим. После проверки распределения атомов по соседям для структуры NiNb  получено (рис. \ref{Ni}), что  у Ni-Nb пики для Ni на 12 и 13 соседях (более 36\% атомов Ni с 12 соседями), для Nb - на 15-16 (более трети атомов с 15 соседями). В результате в сплаве Ni-Nb  уже Nb не образует икосаэдрических кластеров, при это более 12\% многогранников Вороного, построенных вокруг атомов Ni, относятся к виду (0,12,0). Аналогичная картина наблюдается в случае  Ni-Zr  - у Zr в среднем порядка 15-16 соседей и он не образует икосаэдры. У Ni - 12-13 соседей, при этом порядка 7\% из них образуют икосаэдрические кластеры. Таким образом, можно сделать вывод, что отсутствие икосаэдрических кластеров, образованных атомами Zr вызвано стерическими факторами.
\begin{figure}[h!]
	\begin{minipage}[h]{0.5\linewidth}
		\center{\includegraphics[width=1\linewidth]{fig6}} a \\
	\end{minipage}
	\hfill
	\begin{minipage}[h]{0.5\linewidth}
		\center{\includegraphics[width=1\linewidth]{fig7}} \\б
	\end{minipage}
	\vfill
	\begin{minipage}[h]{0.5\linewidth}
		\center{\includegraphics[width=1\linewidth]{fig8}} в \\
	\end{minipage}
	\hfill
	\begin{minipage}[h]{0.5\linewidth}
		\center{\includegraphics[width=1\linewidth]{fig9}} г \\
	\end{minipage}
	\caption{Распределение по числу соседей в сплавах Ni-Zr и Ni-Nb (сверху) и соответствующие им ПКФ (снизу) (сплав Ni-Nb с 30\% Ni, Ni-Zr с 30\% Zr).}
	\label{Ni}
\end{figure}

Для многогранников (0,12,0) была изучена зависимость их числа  от скорости охлаждения  и процентного содержания Nb (рис. \ref{Vp-p}). Первая зависимость  монотонная - процентное содержание таких многогранников снижается с ростом скорости охлаждения. Это объясняется с учетом кинетического характера аморфизации - структура не успевает перестроится из-за высокой скорости охлаждения. Зависимость числа данных многогранников от процентного содержания Nb немонотонна. В начале наблюдается рост, пик в районе 35\% , после чего линейное падение. Это можно объяснить следующим образом. В начале преобладает тот факт, что чем больше Nb, тем «аморфнее» соединение (это рассмотрено в пункте 6), поэтому многогранников больше. Но, чем больше Nb, тем в среднем у Nb больше соседей. Поэтому постепенно сокращается количество кластеров, соответствующих 12 соседям, в частности (0,12,0). 
\begin{figure}[h!]
	\begin{center}
		\includegraphics[scale= 0.65]{fig10}{}{}{}
		\caption{Зависимость числа многогранников (0,12,0) от процентного содержания Nb для различных скоростей охлаждения.}
		\label{Vp-p}
	\end{center}
\end{figure}

Для проверки данного предположения были  рассмотрены зависимости числа атомов Nb с 12, 13 и  14 соседями от процентного содержания Nb. Они описываются линейной зависимостью, с 12 атомами - убывающей, с 14 атомами - возрастающей. При этом общее число атомов с 12 или 14 соседями почти постоянное и не зависит от процентного содержания Nb (рис. \ref{sosedi_pr}). Число атомов Nb с 13 соседями также почти постоянное. Это согласуется с предположением о росте числа соседей из-за стерического фактора и перестроением икосаэрических кластеров в кластеры с 14 соседями.
\begin{figure}[h!]
	\begin{center}
		\includegraphics[scale= 0.6]{fig11}{}{}{}
		\caption{Зависимость числа атомов с 12, 13 и  14 соседями, cуммарного числа атомов с 12 и 14 соседями от процентного содержания Nb.}
		\label{sosedi_pr}
	\end{center}
\end{figure}

Подобные переходы широко обсуждаются в литературе. Икосаэдрические кластеры перестраиваются и видоизменяются с целью увеличения плотности и занятого объема, поскольку это выгодно по энергии \cite{Sheng}.  Также эти переходы  подтверждаются  зависимостями различных многогранников Вороного от температуры (с 12, 13 и 14 гранями). При росте процентного содержания увеличивается число многогранников с 13-14 гранями, а с 12, наоборот, уменьшается. 

Наиболее явно эти переходы отражены в зависимостях (0,12,0) и (2,8,4) от процентного содержания Nb. Эти зависимости очень схожи с зависимостями числа атомов с 12 и 14 соседями соответственно. 
Таким образом, действительно можно говорить о том, что происходит переход между этими 2 типами многогранников, и постепенно икосаэдрические кластеры заменяются на кластеры, соответствующие многогранникам вида (2,8,4). При увеличении скорости охлаждения повышается число атомов Nb с 12 и 13 соседями, что объясняется также кинетическим характером аморфизации и тем, что структура не успевает перестроиться в более стабильную. 

Также были проанализированы зависимости еще нескольких основных многогранников, образуемых Nb, от процентного содержания Nb (рис. \ref{polyhedra_Zr_Nb}б). При увеличении процентного содержания Nb наблюдается рост многогранников вида (3,6,4) и (1,10,2). Помимо того, что это можно объяснить увеличением числа соседей для атомов Nb, это связано c тем, что эти многогранники образуются из многогранников (0,12,0) небольшим количеством перестроений. Таким образом, эти многогранники тоже соответствуют икосаэдрическому окружению, а также наиболее вероятному числу соседей при таком соотношении атомных радиусов. В случае икосаэдров с центром в Zr (рис. \ref{polyhedra_Zr_Nb}а) общая тенденция заключается в том, что почти все кривые меняются очень плавно, не наблюдается сильного роста, как в случае с многогранниками, образованными Nb. Стоит также отметить рост числа многогранников вида (2,8,5) и (2,8,4) отвечающих большому числу соседей, а также сокращение числа многогранников вида (3,6,4), объясняемое снижением числа атомов с 13 соседями. Результаты, полученные в данном исследовании по структуре сплава для различного процентного содержания Nb находятся в хорошем согласии с работой \cite{Reddy}.
\begin{figure}[h!]
	\begin{center}
		\begin{minipage}[h]{0.7\linewidth}
			\center{\includegraphics[width=1\linewidth]{fig12}} \\a 
		\end{minipage}
		\hfill
		\begin{minipage}[h]{0.7\linewidth}
			\center{\includegraphics[width=1\linewidth]{fig13}} \\б
		\end{minipage}
		\caption{Зависимости числа различных многогранников Вороного, образованных  a) Zr и б) Nb соответственно, от процентного содержания Nb.}
		\label{polyhedra_Zr_Nb}
	\end{center}
\end{figure}

\subsection{Анализ ПКФ Zr-Nb, связь ПКФ и икосаэдрического окружения}
В работе проанализированы  три типа ПКФ - Zr-Zr, Zr-Nb и Nb-Nb (рис. \ref{RDF_full}). Можно отметить, что на ПКФ Zr-Zr (рис. \ref{RDF_full}а) отсутствует расщепление второго пика. Вместо него наблюдается плавно спадающий второй пик. Также между первым и вторым пиком выделяется достаточно высокий подпик, не характерный для ПКФ аморфных сплавов. При этом перекрестная ПКФ Zr-Nb (рис. \ref{RDF_full}б)  выглядит типично для аморфных соединений - наблюдается характерное расщепление второго пика ПКФ.
У ПКФ Nb-Nb (рис. \ref{RDF_full}в)  также существуют некоторые особенности - вместо цельного второго пика наблюдается подпик, широкое плато после него и очень слабый второй подпик. 

\begin{figure}[h!]
	\begin{minipage}[h]{0.32\linewidth}
		\center{\includegraphics[width=1\linewidth]{fig14}} a \\
	\end{minipage}
	\hfill
	\begin{minipage}[h]{0.32\linewidth}
		\center{\includegraphics[width=1\linewidth]{fig15}} \\б
	\end{minipage}
	\hfill
	\begin{minipage}[h]{0.32\linewidth}
		\center{\includegraphics[width=1\linewidth]{fig16}} в \\
	\end{minipage}
	\caption{ПКФ а) Zr-Zr, б) Zr-Nb и в) Nb-Nb (50\% Nb).}
	\label{RDF_full}
\end{figure}


Таким образом, на ПКФ Zr-Zr и Nb-Nb в аморфном сплаве Zr-Nb не наблюдается обычное  для аморфных металлов расщепление второго пика. При этом нужно отметить, что расщепление второго пика ПКФ  Zr-Nb действительно является критерием аморфизации, поскольку оно наблюдается только в аморфной фазе. 

В случае ПКФ Ni-Nb картина выглядит похожим образом  - расщепление второго пика наблюдается на ПКФ Ni-Nb, а также на ПКФ Ni-Nb, на ПКФ Nb-Nb оно отсутствует (рис. \ref{Ni}в). 
ПКФ для случая сплава Ni-Zr имеет достаточно нехарактерный вид, что  может быть обусловлено большой разницей в радиусах атомов (рис. \ref{Ni}г) . При этом, в обоих этих сплавах именно атом Ni образует икосаэдрическое окружение, из чего можно сделать предположение о существовании связи между икосаэдрическим окружением атомов и расщеплением второго пика ПКФ

Для нахождения связи между этими двумя критериями из расчетной ячейки были выделены атомы с икосаэдрическим окружением, то есть атомы с 12 ближайшими атомами,  и соседние с ними атомы. Для получившегося набора частиц можно отметить, что количество атомов, задействованных в образовании икосаэдрической структуры велико - более 45\% атомов (c учетом того, что икосаэдрическое окружение характерно только для 6\% атомов и что в рассматривались только многогранники вида (0,12,0) без 
учета деформированных икосаэдров). Следовательно, в образовании икосаэдров задействована почти половина атомов вещества. Также  эти икосаэдры образуют основу структуры аморфного соединения. Они равномерно распространены по всей расчетной ячейке, а атомы, не являющиеся частью икосаэдров, заполняют оставшиеся в икосаэдрической структуре полости. Заполнения всей структуры данными кластерами не происходит, поскольку икосаэдрическая упаковка не является плотнейшей. Поэтому икосаэдрическая структура не повторяется  и не заполняет все пространство \cite{Hoare}. Также в структуре  практически отсутствуют изолированные икосаэдры. Все икосаэдры связаны друг с другом, причем практически всегда кластеры взаимопроникающие, то есть связь очень сильная - по средством семи обобщенных атомов (рис. \ref{icosahedra_ex}б). В этом случае необходимо разорвать 15 связей для отделения икосаэдра от всей остальной  структуры. Преобладание именно такого типа связи подтверждается следующими фактами:
\begin{enumerate}
	\item В структуре с 40\%  содержанием Nb отношение атомов, задействованных в икосаэдрах к атомам с икосаэдрическим окружением  $N_{at}/N_{ico}\approx 7,8$;
	\item В изолированном кластере $N_{at}/N_{ico}=13$  (вершины кластера и его центр);
	\item Для структуры, состоящей только из взаимопроникающих кластеров, справедливо соотношение $N_{at} = 6*N_{ico}+7$. То есть при большом числе икосаэдров в такой структуре отношение атомов,  задействованных в икосаэдрах, к атомам с икосаэдрическим окружением стремится к 6;
	\item В случае связи по вершинам  $N_{at}/N_{ico}=12$, связи по ребру - $11$, связи по грани - $10$. 
\end{enumerate}

Таким образом, полученная  структура образована наиболее крепкими типами связи - каждый  икосаэдрический кластер сильно связан с икосаэдрической структурой в целом. Также нужно отметить, что атомы Zr и Nb  в равной степени участвуют в образовании икосаэдрического окружения - в выделенной структуре процентное содержание атомов  Zr и Nb практически такое же, как и в изначальной - 59\% Zr и 41\% Nb.

Отметим основные особенности ПКФ  икосаэдрической структуры ($\text{ПКФ}_{ico}$), которая обладает рядом отличий по сравнению с ПКФ структуры в целом (рис. \ref{RDF_full}, рис. \ref{RDF_ico_besico}). Для нее характерно  расщепление второго пика ($\text{П}_2$) на ПКФ Nb-Nb.  При этом, на ПКФ атомов, не задействованных в икосаэдрах ($\text{ПКФ}_{ooico}$),  характерное расщепление $\text{П}_2$ на ПКФ Zr-Nb (рис. \ref{RDF_ico_besico}г) отсутствует, а на ПКФ Nb-Nb (рис. \ref{RDF_ico_besico}е)   выделяется широкий второй пик. При этом,  ПКФ Zr-Zr больше соответствует стандартному виду ПКФ аморфных металлов с тем  отличием, что вместо второго подпика второго пика ($\text{П}_2^2$)  выделяется широкое плато после первого подпика второго пика ($\text{П}_2^1$) .На ПКФ Zr-Zr (рис. \ref{RDF_ico_besico}а) $\text{П}_2$ значительно уже, чем на ПКФ Zr-Zr всех атомов аморфной структуры (рис. \ref{RDF_full}а) и на ПКФ атомов, лежащих вне икосаэдрических кластеров (рис. \ref{RDF_ico_besico}б). Наконец, если сравнить полные ПКФ атомов икосаэдрической структуры и всех атомов аморфного сплава (рис. \ref{RDF_full_ico}), можно заметить, что расщепление второго пика наблюдается значительно сильнее в случае атомов икосаэдрической структуры. Таким образом, можно сделать вывод, что расщепление $\text{П}_2$, которое наблюдается  на ПКФ, обусловлено именно образовавшейся в веществе икосаэдрической структурой. 
\begin{figure}[h!]
	\begin{minipage}[h!]{0.47\linewidth}
		\center{\includegraphics[width=1\linewidth]{fig17}} a \\
	\end{minipage}
	\hfill
	\begin{minipage}[h!]{0.47\linewidth}
		\center{\includegraphics[width=1\linewidth]{fig18}} б \\
	\end{minipage}
	\vfill
	\begin{minipage}[h!]{0.47\linewidth}
		\center{\includegraphics[width=1\linewidth]{fig19}} \\в
	\end{minipage}
	\hfill
	\begin{minipage}[h!]{0.47\linewidth}
		\center{\includegraphics[width=1\linewidth]{fig20}} г \\
	\end{minipage}
	\vfill
	\begin{minipage}[h!]{0.47\linewidth}
		\center{\includegraphics[width=1\linewidth]{fig21}} \\д
	\end{minipage}
	\hfill
	\begin{minipage}[h!]{0.47\linewidth}
		\center{\includegraphics[width=1\linewidth]{fig22}} е \\
	\end{minipage}
	\caption{ПКФ Zr-Zr, Zr-Nb и Nb-Nb для атомов, содержащихся в кластерах (а), в) и д), соответственно), и атомов вне кластеров (б),г) и е), соответственно).}
	\label{RDF_ico_besico}
\end{figure}

\begin{figure}[h!]
	\begin{center}
		\includegraphics[scale= 0.6]{fig23}{}{}{}
		\caption{Полная ПКФ атомов в икосаэдрических кластеров и атомов всей аморфной структуры.}
		\label{RDF_full_ico}
	\end{center}
\end{figure}
Для объяснения полученной $\text{ПКФ}_{ico}$ были проанализированы отдельные икосаэдры и взаимопроникающие кластеры.  Геометрическое рассмотрение идеального икосаэдрического кластера и двух взаимопроникающих кластеров приведено в  работе \cite{Liang}. Расстояния, которые лежат в области первого пика ПКФ ($\text{П}_1$), соответствуют расстоянию между ближайшими соседями, $\text{П}_2^1$ соответствует расстояние между двумя атомами в одном пятиугольнике, разделенные третьим атомом (рис. \ref{icosahedra_ex}а, атомы 3 и 4). В случае $\text{П}_2^2$ уже нельзя выбрать конкретные атомы, соответствующие ему, поскольку на относительное расположение атомов при таких расстояниях сильно влияет большое количество атомов окружения. Рассмотрим более подробно кластеры, приводя для каждого интересующего отрезка ПКФ атомы, расстояние между которыми соответствуют этому отрезку. 
\begin{figure}[h!]
	\begin{minipage}[h]{0.5\linewidth}
		\center{\includegraphics[width=1\linewidth]{fig3}} a \\
	\end{minipage}
	\hfill
	\begin{minipage}[h]{0.5\linewidth}
		\center{\includegraphics[width=1\linewidth]{fig4}} б\\
	\end{minipage}
	\caption{ Пример а)  икосаэдрического кластера; б) двух взаимопроникающих кластеров (серые - атомы Nb, черные - Zr).}
	\label{icosahedra_ex}
\end{figure}
В начале была рассмотрена ПКФ Zr-Zr. Первый  участок - высокая ступень после $\text{П}_1$. Этой ступени соответствует расстояние между вершиной икосаэдра и атомом, лежащим в ближайшем пятиугольнике (рис. \ref{icosahedra_ex}а, 1 и 5 атомы). В случае ПКФ Nb-Nb аналогичное расстояние между атомами дает вершину, лежащую внутри $\text{П}_1$. Объяснением этому служит стерический фактор. К примеру, в икосаэдре, изображенном на рис.\ref{icosahedra_ex} а видно, что эта связь находится в икосаэдре  с еще одним атомом Zr, имеющем связи с обоими отмеченными атомами. В результате этого связь становится длиннее, чем в случае ПКФ Nb-Nb.  Аналогичные комбинации были обнаружены еще в нескольких кластерах, там они тоже дают ступень после $\text{П}_1$. 

После этого было исследовано, с чем связано отсутствие расщепления $\text{П}_2$ Zr-Zr  $\text{ПКФ}_{ico}$ .  Первое, что нужно отметить - $\text{ПКФ}_{ico}$ Zr-Zr очень  плавно уменьшается после второго пика и там, где в случае Nb различим отдельный $\text{П}_2^2$, наблюдается широкое плато. Для интерпретации этого участка $\text{ПКФ}_{ico}$ была рассмотрена система взаимопроникающих кластеров. Такая система представлена на рис. \ref{icosahedra_ex}б. На ней выделены два атома, расстояние между которыми соответствует рассматриваемому участку -  расстояние между атомами, лежащими в пятиугольниках $C$ и $E$ через один пятиугольник $D$ (рис. \ref{RDF_ico_besico}а, рис. \ref{icosahedra_ex}б 6 и 7 атомы). При этом данное расстояние определяется очень большим количеством атомов, что и может в результате давать широкий разброс по расстоянию, и, как следствие, широкое плато вместо одиночного пика. Еще одна пара атомов, расстояние между которыми соответствует плато после $\text{П}_2$, это расстояние между вершинами пятиугольников $A$ и $B$, соединяющий вектор которых проходит через центр пятиугольника (рис. \ref{RDF_ico_besico}а, рис. \ref{icosahedra_ex}а атомы 3 и 5). Также существует еще одна комбинация - вершина икосаэдра и атом во втором от него пятиугольнике $D$ (рис. \ref{RDF_ico_besico}а, рис. \ref{icosahedra_ex}б 8 и 9 атомы). В случае увеличения размера икосаэдров в сплаве с 30\% Nb плато после $\text{П}_2^1$ соответствует расстоянию между двумя атомами в  одном пятиугольнике $B$ через один промежуточный (рис.\ref{RDF_ico_besico}а, рис. \ref{icosahedra_ex}а 3 и 4 атомы).  

Последний рассматриваемый участок $\text{ПКФ}_{ico}$ Zr-Zr - $\text{П}_2^1$. По теоретическим расчетам, проведенным для правильного икосаэдра, этот пик соответствует расстоянию между двумя атомами в пятиугольнике, разделенными промежуточным атомом. Это действительно наблюдается  (рис.\ref{RDF_ico_besico}а, рис.\ref{icosahedra_ex}а 3 и 4 атомы). На практике, был обнаружен еще один возможный вариант - два атома в соседних пятиугольниках $D$ и $E$ , повернутые на один атом друг относительно друга (рис.\ref{RDF_ico_besico}а, рис. \ref{icosahedra_ex}б 7 и 9 атомы).

Также данный подход был использован для объяснения $\text{ПКФ}_{ico}$ Nb-Nb. Некоторые пики соответствуют тем же расстояниям что и раньше -  первый пик $\text{ПКФ}_{ico}$ образован расстоянием между ближайшими соседями (рис.\ref{RDF_ico_besico}д, рис.\ref{icosahedra_ex} 6 и 12 атомы). $\text{П}_2^1$ $\text{ПКФ}_{ico}$ образован расстоянием между двумя атомами в пятиугольнике, разделенными промежуточным атомом (рис. \ref{RDF_ico_besico}д, рис. \ref{icosahedra_ex}а 3 и 4 атомы). Для исследования расстояний, которые соответствуют  $\text{П}_2^2$, была рассмотрена система из нескольких икосаэдров. Этому подпику соответствует расстояние между  атомами в пятиугольниках, разделенных одним пятиугольником (рис. \ref{RDF_ico_besico}д, рис. \ref{icosahedra_ex}б 7 и 12 атомы, пятиугольники $C$ и $E$). Возможна также еще одна комбинация, аналогичная предыдущей - тоже два атома в пятиугольниках, разделенных одним пятиугольником,  только атомы уже не ближайшие, а повернутые на один друг относительно друга (рис. \ref{RDF_ico_besico}д, рис. \ref{icosahedra_ex}б  6 и 7 атомы в пятиугольниках $C$ и $E$). Наконец, последний вариант - расстояние между центром одного икосаэдра и самой дальней вершиной второго икосаэдра (по сути, противоположные вершины одного икосаэдра (рис. \ref{RDF_ico_besico}д, рис. \ref{icosahedra_ex}б 10 и 11 атомы). 


Как можно заметить, количество возможных вариантов больше, чем в случае $\text{ПКФ}_{ico}$ Zr-Zr за счет атомов, лежащих в центре икосаэдров. Поэтому и $\text{П}_2^2$ в случае  Nb-Nb явно выражен, в отличие от $\text{ПКФ}_{ico}$ Zr-Zr - он реализуется большим числом способов, и, соответственно, чаще встречается в структуре. Для $\text{ПКФ}_{ico}$ Zr-Nb все пики соответствуют разобранным выше комбинациям атомов.


Таким образом, в работе полностью был объяснен вид всех  ПКФ - Zr-Zr, Nb-Nb и Zr-Nb. В частности, было показано, что расщепление второго пика ПКФ вызвано  икосаэдрической подструктурой, образующейся в аморфном сплаве. Эта структура состоит из взаимопроникающих кластеров, заполняющих весь объем вещества.

\section{Температура стеклования}
В пункте 2 были описаны косвенные критерии стеклования, использованные в работе. Определяемая с их помощью температура стеклования зависит от того, какие изменения берутся за основу. К примеру, в экспериментах зачастую используют изменения коэффициентов теплоемкости \cite{tropin2015heat} и вязкости \cite{konstantinova2009kinematicheskaya}, а также изменение коэффициента Холла \cite{Kuzmenko}. Далее рассмотрены структурные критерии стеклования,  результаты определения температуры стеклования и их сравнение друг с другом.

\subsection{Зависимость числа икосаэдрических кластеров от температуры}
Для сравнения были рассмотрены  зависимости  числа икосаэдрических кластеров от температуры при охлаждении расплава и при плавлении аморфной фазы, а также зависимости числа других многогранников для тех же процессов. На  рис.\ref{0120_284}  приведены зависимости многогранников вида (0,12,0) и  (2,8,4) в качестве примера. Для многогранников (2,8,4) можно отметить, что данные, полученные при нагревании и на охлаждении практически идентичны, процессы проходили по одному пути. Также на графике отсутствуют какие-либо резкие перегибы и переломы на зависимости числа многогранников от температуры, то есть зависимости числа таких многогранников в жидкости и в аморфной фазе одинаковы и разграничить эти фазы по данной зависимости не представляется возможным.
\begin{figure}[h!]
	\begin{center}
		\begin{minipage}[h]{0.6\linewidth}
			\center{\includegraphics[width=1\linewidth]{fig24}} \\a 
		\end{minipage}
		\hfill
		\begin{minipage}[h]{0.6\linewidth}
			\center{\includegraphics[width=1\linewidth]{fig25}} \\б
		\end{minipage}
		\caption{а) Зависимость числа многогранников (2,8,4)  от температуры при охлаждении расплава Zr-Nb и плавлении аморфной фазы Zr-Nb, 55\% Nb. 
			б)Зависимость числа многогранников (0,12,0) от температуры при охлаждении расплава Zr-Nb и плавлении аморфной фазы Zr-Nb, 55\% Nb. Сплошная кривая - аппроксимация функции $N_{ic}(T)$ при плавлении сигмоидальной функцией. На вставке - вторая производная функции количества многогранников (0,12,0) от температуры.}
		\label{0120_284}
	\end{center}
\end{figure}
При этом на зависимости числа икосаэдрических кластеров от температуры наблюдается гистерезис, причем он тем сильнее выражен, чем глубже находится рассматриваемый сплав в области существования аморфной фазы на фазовой диаграмме. Также на зависимости, соответствующей нагреванию аморфного сплава, можно отметить три характерных участка.

При  увеличении  температуры сначала наблюдается участок, соответствующий аморфной фазе, после этого участок, на котором происходит  резкое уменьшение числа икосаэдрических кластеров, а в конце  происходит переход в область жидкости, где число икосаэдров при росте температуры меняется уже не так резко (снижается чуть более чем на процент при изменении температуры на $\approx 600\text{ K}$). Таким образом, возможно выделить  диапазон температур, соответствующий переходной фазе между жидкостью и стеклом. В этом переходном состоянии идет разрушение икосаэдрических кластеров, и, как следствие, разрушение аморфной структуры вещества.

На приведенном рис. \ref{0120_284}б  показаны рассчитанные значения числа  икосаэдров при различных температурах. Данная кривая лучше всего аппроксимируется сигмоидальной зависимостью:
\begin{equation}
N_{ico} = \frac{A}{1+e^{-k(T-b)}}
\label{eq_3}
\end{equation} 
Для нахождения точек изменения зависимости икосаэдров от температуры были найдены точки экстремума второй производной аппроксимирующей функции:
\begin{equation}
\frac{d^2N_{ico}}{dT^2} = \frac{2Ak^2e^{-2\cdot k(T-b)}}{(1 + e^{-k(T-b)})^3}-\frac{k^2e^{-k\cdot(T-b)}}{(1 + e^{-k(T-b)})^2 }
\label{eq_4}
\end{equation} 
Полученные температуры соответствуют температурам $T_1$ и $T_2$, разделяющих график на три характерных участка.  Также по графику второй производной можно отметить, что на промежутке температур $T_1$ - $T_2$ скорость изменения числа многогранников максимальна.
Необходимо сказать, что диапазон температур, отвечающих переходному состоянию, очень широк - от  $1100 K$ до $1850 K$. Температурой стеклования выбрана температура, при которой начинается плавление аморфной фазы, поскольку именно при данной температуре исчезает расщепление второго пика, являющееся критерием аморфной фазы.  Наконец, нужно отметить, что определить температуры изменения зависимостей можно только на плавлении стекла. При охлаждении не наблюдается перелома, соответствующего стеклованию сплава (рис. \ref{0120_284}б,  Табл. \ref{tab:my-table}). Данная особенность, наблюдаемая при исследовании перехода  <<жидкость-стекло>>  с помощью различных структурных и термодинамических характеристик, широко рассматривается в литературе. Теоретически данная особенность объясняется в \cite{tropin2015heat}, \cite{tropin2011dependence}  с помощью параметра структурного порядка. В экспериментальных работах (к примеру, \cite{aji2015kinetic}) явление гистерезиса в переходах <<жидкость-стекло>>  и  <<стекло-жидкость>>  объясняется следующим образом.  Переход из стабильного состояния (жидкости) в метастабильное состояние (стекло) при температурах, меньших чем температура стеклования, происходит самопроизвольно, то есть при фиксации температуры ниже $T_g$ процесс образования аморфной фазы продолжается. Переход из стекла в жидкость в то же время гораздо больше похож на фазовые переходы второго рода \cite{Dandar}, процесс не идет самопроизвольно.

\subsection{Критерий Вендта-Абрахама}
Характерный вид зависимости отношения первого минимума к первому максимуму $g_{min}/g_{max}$ ПКФ от температуры, с помощью которого определялась температура стеклования по критерию Вендта-Абрахама, показан на рис. \ref{VA_criterium}.  
\begin{figure}[h!]
	\begin{center}
		\includegraphics[scale= 0.6]{fig2}{}{}{}
		\caption{Зависимость отношения первого минимума к первому максимуму ПКФ от температуры, 55\% Nb.}
		\label{VA_criterium}
	\end{center}
\end{figure}
Основные особенности данной зависимости заключаются в следующем. Во-первых, наблюдается гистерезис - различие кривых  охлаждения жидкости и плавления аморфного сплава. Это является еще одним доказательством связи вида ПКФ и икосаэдрического окружения атомов - подобное различие поведения числа многогранников от температуры между охлаждением и нагреванием наблюдается исключительно для многогранников (0,12,0), а значит, именно ими и определяется вид ПКФ. Во-вторых, определяемая в данном случае температура очень сильно зависит от того, каким образом решено разбить график на две прямых. Поэтому точность данного метода не является  высокой. Возвращаясь к предыдущему подпункту, можно сказать, что температура стеклования, определенная по данному критерию, соответствует  температуре, разделяющей аморфную и переходные фазы (Табл. \ref{tab:my-table}).

\subsection{Площадь под вторым пиком ПКФ}
Характерный вид зависимости площади под двумя разделенными подпиками второго пика ПКФ представлен на рис. \ref{Square_criterium}.
\begin{figure}[h!]
	\begin{center}
		\includegraphics[scale= 0.6]{fig26}{}{}{}
		\caption{Зависимость площади под вторым пиком ПКФ Zr-Nb от температуры, 50\% Nb. ($S_{max}$  соответствует наибольшей площади под вторым пиком ПКФ).}
		\label{Square_criterium}
	\end{center}
\end{figure}

На ней наблюдается, как и в случаях с зависимостью числа икосаэдрических кластеров от температуры и критерия Вендта-Абрахама, гистерезис - при охлаждении и нагревании кривые различаются, как и температуры, определяемые этим методом. При этом для данного критерия скачок площади, соответствующий температуре стеклования, явный. Поэтому температура определяется гораздо точнее, чем в двух предыдущих методах. В целом, этот метод показал хорошее согласие с двумя другими - температура стеклования, определяемая по зависимости площади от температуры, совпадает с двумя другими структурными методами (Табл. \ref{tab:my-table}). Особым достоинством данного метода является возможность определения температуры стеклования при охлаждении,  в отличие от двух других структурных методов. Тем не менее, нужно отметить, что для исключения влияния погрешностей расчета ПКФ на точность определения температуры стеклования, необходимо использовать систему с большим количеством атомов в расчетной ячейке (примерно $10^5$). 


\begin{table}[h!]
	\centering
	\caption{Таблица температур стеклования для скорости $11\cdot10^{12}  \text{ K/пс}$.}
	\label{tab:my-table}
	\resizebox{\textwidth}{!}{%
		\begin{tabular}{|c|c|c|c|c|c|c|}
			\hline
			& \multicolumn{2}{|c|}{$T$ \textit {по площади под вторым пиком,} $K$,} & $T$ \textit{по критерию Вендта-Абрахама,} $K$, & $T$ \textit{по икосаэдрическим кластерам,}$K$,  \\ 
			& \multicolumn{2}{|c|} {$\Delta T = 10K$}& \multicolumn{1}{|c|} {$\Delta T = 50K$} & \multicolumn{1}{|c|} {$\Delta T = 50K$} \\ \hline
			\%, Nb & охлаждение & нагревание  & \multicolumn{1}{|c|}{нагревание} & \multicolumn{1}{|c|}{нагревание}\\ \hline
			25 & 1100 & 1050  & 1000 & 1000 \\ \hline
			27,5 & 1100 & 1050  & 1000 & 1000 \\ \hline
			30 & 1050 & 1100  & 1200 & 1100 \\ \hline
			35 & 1000 & 1100  & 1150 & 1150 \\ \hline
			40 & 1200 & 1250  & 1200 & 1200 \\ \hline
			45 & 1150 & 1050  & 1200 & 1100 \\ \hline
			50 & 1100 & 1050  & 1200 & 1200 \\ \hline
			55 & 1000 & 1100  & 1200 & 1200 \\ \hline
			65 & 1000 & 1100  & 1100 & 1100 \\ \hline
			75 & 900 & 800  & 1000 & 1000 \\ \hline
		\end{tabular}%
	}
\end{table}
\section{Фазовая диаграмма}
Проведенные расчеты по стеклованию расплава Zr-Nb для различного процентного содержания и различной скорости охлаждения позволили получить фазовую диаграмму состояний Zr-Nb в координатах «Процентное содержание Nb - скорость охлаждения» (рис. \ref{K(T)}). На ней выделены три области, соответствующие трем различным состояниям. Центральная область соответствует аморфному сплаву, внешняя область - нанокристаллическому соединению, между ними - переходное состояние, в котором уже наблюдаются зародыши кристаллов, но структура вещества в целом еще аморфна. Минимальная скорость, при которой является возможным получить аморфный сплав Zr-Nb заданного процентного содержания, называется критической скоростью охлаждения $K$. 
\begin{figure}[h!]
	\begin{center}
		\includegraphics[scale= 0.15]{fig28}{}{}{}
		\caption{Зависимость критической скорости охлаждения Zr-Nb от процентного содержания Nb.}
		\label{K(T)}
	\end{center}
\end{figure}
Важным выводом из полученной фазовой диаграммы  является то, что минимум критической скорости не соответствует максимальному процентному содержанию икосаэдрических кластеров, которые являются основными структурными единицами металлического стекла. Для зависимости процентного содержания икосаэдрических кластеров от процентного содержания Nb максимум соответствует примерно 35\% Nb, а для критической скорости минимум приходится на 45-50\% Nb. 

\begin{figure}[h!]
	\begin{center}
		\includegraphics[scale= 0.3]{fig27}{}{}{}
		\caption{Зависимости отношения числа атомов в икосаэдрических кластерах в сплаве к числу икосаэдрических  кластеров от процентного содержания Nb и числа кластеров от процентного содержания Nb.}
		\label{icocenterofico}
	\end{center}
\end{figure}
Возможным объяснением такого отклонения может служить зависимость $N_{at}/N_{ico}$ от процентного содержание Nb, рассмотренная  в пункте 3.1. По этой зависимости видно (рис. \ref{icocenterofico}), что ее минимум соответствует как раз процентному содержанию Nb около 45-50\%. Чем меньше это отношение, тем крепче связь между икосаэдрами. При этом нельзя говорить о том, что наблюдается зависимость только от числа икосаэдров в сплаве. Как видно из рис. \ref{icocenterofico}, максимум числа икосаэдров наблюдается при 65\% Nb в сплаве, что не соответствует минимальной критической скорости. Таким образом, можно сделать вывод, что критическая скорость зависит не от того, насколько часто икосаэдры образуются в системе и не от их числа в системе,  а  от их связи друг с другом. Чем эта связь крепче, тем ниже  критическая скорость стеклования.






%\chapter*{Выводы}
\chapter*{Выводы}
\addcontentsline{toc}{chapter}{Выводы}
В работе проведено моделирование стеклования сплава Zr-Nb, исследована аморфная фаза данного сплава, ее строение и критические скорости охлаждения. Установлено, что изменение температурной зависимости числа икосаэдрических кластеров ($N_{ico}$) для сплава Zr-Nb может быть критерием стеклования. Также установлено, что расщепление второго пика перекрестной ПКФ Zr-Nb для системы Zr-Nb может являться критерием стеклования. Показана связь между расщеплением второго пика ПКФ и икосаэдрическим окружением атомов. Расщепление второго пика ПКФ объяснено через конкретные расстояния между атомами в системе взаимопроникающих кластеров.

Исследовано влияние содержания Nb и скорости охлаждения ($K$) на число многогранников Вороного. Предложено объяснение, основанное на  стерическом факторе - различии в размерах атомов Zr и Nb. Из-за большего размера Zr у него в среднем более чем 12 соседей и поэтому очень редко возможно образование кластеров с Zr в центре. Показано, что структура, состоящая  из икосаэдрических кластеров, соединенных по грани, или из взаимопроникающих кластеров, является основной для аморфного сплава Zr-Nb. Она заполняет весь объем сплава, определяет его структуру и условия формирования.  

Определен температурный диапазон стеклования по  зависимости $N_{ico}(T)$. Нижняя граница этого диапазона  совпадает с температурой появления  расщепления второго пика ПКФ. Эта температура согласуется  с температурой стеклования, определенной по критерию Вендта-Абрахама, в пределах погрешности. Также определена температура стеклования по площади под вторым пиком ПКФ, совпадающая с двумя другими критериями.  


Получена зависимость пороговой скорости $K$ от процентного содержания Nb для сплава Zr-Nb. Показано, что поведение этой зависимости совпадает с зависимостью отношения числа атомов в икосаэдрических кластерах к числу центров икосаэдров от процентного содержание Nb.
\begin{thebibliography}{44}
	\bibitem{Zolotukhin}
	Золотухин И. В., Калинин Ю. Е. Аморфные металлические сплавы //Успехи физических наук. – 1990. – Т. 160. – №. 9. – С. 75-110.
	
	\bibitem{Egami}
	Egami T., Srolovitz D. Local structural fluctuations in amorphous and liquid metals: a simple theory of the glass transition //Journal of Physics F: Metal Physics. – 1982. – Т. 12. – №. 10. – С. 2141.
	
	\bibitem{Gaskell}
	Gaskell P. H. Local and medium range structures in amorphous alloys //Journal of Non-Crystalline Solids. – 1985. – Т. 75. – №. 1-3. – С. 329-346.
	
	\bibitem{Waseda}
	Waseda Y., Chen H. S. A structural study of metallic glasses containing boron (Fe-B, Co-B, and Ni-B) //Physica status solidi (a). – 1978. – Т. 49. – №. 1. – С. 387-392.
	
	\bibitem{Pisarev}
	Кирова Е. М., Норман Г. Э., Писарев В. В. Моделирование стеклования тонкого слоя расплава алюминия при сверхбыстром охлаждении в изобарических условиях //Письма в Журнал экспериментальной и теоретической физики. – 2019. – Т. 110. – №. 5. – С. 343-347.
	
	\bibitem{Liu}
	Liu X. J. et al. Metallic liquids and glasses: atomic order and global packing //Physical review letters. – 2010. – Т. 105. – №. 15. – С. 155501.
	
	\bibitem{Kolotova}
	Колотова Л. Н., Норман Г. Э., Писарев В. В. Стеклование переохлажденного расплава алюминия: молекулярно-динамическое исследование //Журнал физической химии. – 2015. – Т. 89. – №. 5. – С. 796-800.
	
	\bibitem{Reddy}
	Reddy K. V., Pal S. Evaluation of glass forming ability of Zr–Nb alloy systems through liquid fragility and Voronoi cluster analysis //Computational Materials Science. – 2019. – Т. 158. – С. 324-332.
	
	\bibitem{Sheng}
	Sheng H. W. et al. Atomic packing and short-to-medium-range order in metallic glasses //Nature. – 2006. – Т. 439. – №. 7075. – С. 419-425.
	
	\bibitem{Hoare}
	Hoare M. Stability and local order in simple amorphous packings //Annals of the New York Academy of Sciences. – 1976. – Т. 279. – №. 1. – С. 186-207.
	
	\bibitem{Wei}
	Wei Y. D. et al. A comparative study on local atomic configurations characterized by cluster-type-index method and Voronoi polyhedron method //Computational Materials Science. – 2016. – Т. 123. – С. 214-223.
	
	\bibitem{Wakeda}
	Wakeda M. et al. Relationship between local geometrical factors and mechanical properties for Cu–Zr amorphous alloys //Intermetallics. – 2007. – Т. 15. – №. 2. – С. 139-144.
	
	\bibitem{Xie}
	Xie Z. C. et al. Evolution of icosahedral clusters during the rapid solidification of liquid TiAl alloy //Physica B: Condensed Matter. – 2014. – Т. 440. – С. 130-137.
	
	\bibitem{Liang}
	Liang Y. C. et al. Influence of icosahedral order on the second peak splitting of pair distribution function for Mg70Zn30 metallic glass //Journal of alloys and compounds. – 2014. – Т. 597. – С. 269-274.
	
	\bibitem{Pan}
	Pan S. P. et al. Origin of splitting of the second peak in the pair-distribution function for metallic glasses //Physical Review B. – 2011. – Т. 84. – №. 9. – С. 092201.
	
	\bibitem{Fukunaga}
	Fukunaga T. et al. Voronoi analysis of the structure of Cu–Zr and Ni–Zr metallic glasses //Intermetallics. – 2006. – Т. 14. – №. 8-9. – С. 893-897.
	
	\bibitem{Evteev}
	Евтеев А. В., Косилов А. Т., Левченко Е. В. Структурная модель стеклования чистых металлов //Письма в Журнал экспериментальной и теоретической физики. – 2002. – Т. 76. – №. 2. – С. 115-117.
	
	\bibitem{Gordeev}
	Гордеев И. С., Стариков С. В., Сравнение различных методов атомистического моделирования для расчета температуры фазового перехода на примере Циркония// Журнал экспериментальной и теоретической физики. – 2019. – Т. 155. – №. 5. – С. 878-885.
	
	\bibitem{Nikulina}
	Никулина А. В. Цирконий-ниобиевые сплавы для элементов активных зон реакторов с водой под давлением //Вопросы атомной науки и техники. Серия: Материаловедение и новые материалы. – 2005. – №. 1. – С. 190-190.
	
	\bibitem{Chernyaeva}
	Черняева Т. П. и др. Особенности структуры закаленных сплавов Zr-Nb //Вопросы атомной науки и техники. – 2011.
	
	\bibitem{Derevyako}
	Дерявко И. И. и др. Реакторная отработка стержневых карбидных твэлов ЯРД и ЯЭДУ //Вестник НЯЦ РК. – 2015. – С. 105.
	
	\bibitem{Smirnova}
	Smirnova D. E., Starikov S. V. An interatomic potential for simulation of Zr-Nb system //Computational Materials Science. – 2017. – Т. 129. – С. 259-272.
	
	\bibitem{lin2013n}
	Lin D. Y. et al. An n-body potential for a Zr–Nb system based on the embedded-atom method //Journal of Physics: Condensed Matter. – 2013. – Т. 25. – №. 10. – С. 105404.
	
	\bibitem{Klement}
	Klement W., Willens R. H., Duwez P. O. L. Non-crystalline structure in solidified gold–silicon alloys //Nature. – 1960. – Т. 187. – №. 4740. – С. 869-870.
	
	\bibitem{Chaudhari}
	Chaudhari P., Cuomo J. J., Gambino R. J. Amorphous metallic films for magneto‐optic applications //Applied Physics Letters. – 1973. – Т. 22. – №. 7. – С. 337-339.
	
	\bibitem{Chadhauri2.0}
	Chaudhari P., Cuomo J. J., Gambino R. J. Amorphous metallic films for bubble domain applications //IBM Journal of Research and Development. – 1973. – Т. 17. – №. 1. – С. 66-68.
	
	\bibitem{Rhyne}
	Rhyne J. J., Schelleng J. H., Koon N. C. Anomalous magnetization of amorphous Tb Fe 2, Gd Fe 2, and Y Fe 2 //Physical Review B. – 1974. – Т. 10. – №. 11. – С. 4672.
	
	\bibitem{Antonets}
	Антонец И. В. и др. Особенности наноструктуры и удельной проводимости тонких пленок различных металлов //Журнал технической физики. – 2004. – Т. 74. – №. 3. – С. 24-27.
	
	\bibitem{Kalin}
	Калин Б. А. Физическое материаловедение Том 5 Материалы с заданными свойствами //М.: МИФИ. – 2008.
	
	\bibitem{Glezer}
	Глезер А., Шурыгина Н. Аморфно-нанокристаллические сплавы. – Litres, 2018.
	
	\bibitem{Abrosimova}
	Абросимова Г. Е. Эволюция структуры аморфных сплавов //Успехи физических наук. – 2011. – Т. 181. – №. 12. – С. 1265-1281.
	
	\bibitem{Danilova}
	Данилова И. И. и др. Производство аморфной и нанокристаллической ленты методом литья на одновалковой МНЛЗ //Вестник Южно-Уральского государственного университета. Серия: Металлургия. – 2008. – №. 9 (109).
	
	\bibitem{Davies}
	Davies H. A., Aucote J., Hull J. B. Amorphous nickel produced by splat quenching //Nature Physical Science. – 1973. – Т. 246. – №. 149. – С. 13-14.
	

	
	\bibitem{Sudzuki}
	Судзуки К., Фудзимори Х., Хасимото К. Аморфные металлы. Металлургия. – 1987.
	
	\bibitem{Prokhorov}
	Прохоров А. М. Физическая энциклопедия. – Рипол Классик, 1988. – Т. 1.
	
	\bibitem{Glezer2002}
	Глезер А. М. Аморфные и нанокристаллические структуры: сходства, различия, взаимные переходы //Российский химический журнал. – 2002. – Т. 46. – №. 5. – С. 57-63.
	
	\bibitem{Chandler}
	Chandler D. Introduction to modern statistical //Mechanics. Oxford University Press, Oxford, UK. – 1987.
	
		\bibitem{Levchenko}
	Левченко Е. В. и др. Кластерная модель структурной организации аморфного железа //Физика металлов и металловедение. – 2010. – Т. 109. – №. 6. – С. 603-607.
	
	\bibitem{Pryadilschikov}
	Прядильщиков А. Ю. и др. Молекулярно-динамическое изучение процесса стеклования бинарного сплава Ni 60 Ag 40 //Журнал экспериментальной и теоретической физики. – 2007. – Т. 132. – №. 6. – С. 1352-1358.	
	
	\bibitem{Medvedev}
	Медведев Н. Н. Метод Вороного-Делоне в исследовании структуры некристаллических систем. – 2000.
	
	\bibitem{Dandar}
	Дандар Д. С. и др. Релаксационные аспекты перехода жидкость—стекло //Успехи физических наук. – 2019. – Т. 189. – №. 2. – С. 113-133.
	
	\bibitem{tropin2015heat}
	Tropin T. V. et al. Heat capacity measurements and modeling of polystyrene glass transition in a wide range of cooling rates //Journal of Non-Crystalline Solids. – 2015. – Т. 409. – С. 63-75.
	
	\bibitem{konstantinova2009kinematicheskaya}	
	Константинова Н. Ю., Попель П. С., Ягодин Д. А. Кинематическая вязкость жидких сплавов медь-алюминий //Теплофизика высоких температур. – 2009. – Т. 47. – №. 3. – С. 354-359.
	
	\bibitem{Kuzmenko}
	Кузьменко В. М. и др. Критические параметры аморфных металлических пленок (обзор) Введение //Украинский физический журнал. – 1976. – Т. 21. – №. 6. – С. 883.
	
	\bibitem{Wendt}
	Wendt H. R., Abraham F. F. Empirical criterion for the glass transition region based on Monte Carlo simulations //Physical Review Letters. – 1978. – Т. 41. – №. 18. – С. 1244.
	
	\bibitem{Hamelberg}
	Hamelberg D., Mongan J., McCammon J. A. Accelerated molecular dynamics: a promising and efficient simulation method for biomolecules //The Journal of chemical physics. – 2004. – Т. 120. – №. 24. – С. 11919-11929.
	
	\bibitem{Lennard}	
	Lennard-Jones J. E. On the forces between atoms and ions //Proceedings of the Royal Society of London. Series A, Containing Papers of a Mathematical and Physical Character. – 1925. – Т. 109. – №. 752. – С. 584-597.
	
	\bibitem{Daw}
	Daw M. S., Baskes M. I. Embedded-atom method: Derivation and application to impurities, surfaces, and other defects in metals //Physical Review B. – 1984. – Т. 29. – №. 12. – С. 6443.
	
	\bibitem{mishin2005phase}
	Mishin Y., Mehl M. J., Papaconstantopoulos D. A. Phase stability in the Fe–Ni system: Investigation by first-principles calculations and atomistic simulations //Acta materialia. – 2005. – Т. 53. – №. 15. – С. 4029-4041.
	
	\bibitem{Plimpton}
	Plimpton S. Fast parallel algorithms for short-range molecular dynamics. – Sandia National Labs., Albuquerque, NM (United States), 1993. – №. SAND-91-1144.
	
	\bibitem{Gilyaev}
	Янилкин А. В. и др. Применение суперкомпьютеров для молекулярно-динамического моделирования процессов в конденсированных средах //Вычислительные методы и программирование. – 2010. – Т. 11. – №. 1. – С. 111-116.
	

	

	
	\bibitem{Stukowski}
	Stukowski A. Visualization and analysis of atomistic simulation data with OVITO–the Open Visualization Tool //Modelling and Simulation in Materials Science and Engineering. – 2009. – Т. 18. – №. 1. – С. 015012.
	
	\bibitem{Tretyakov}
	Третьяков Ю. Д. и др. Неорганическая химия. Химия элементов: Учебник для вузов: в 2 книгах. Кн. 1. М //Химия. – 2001.
	
	\bibitem{Bondarev}
	Бондарев А. В., Батаронов И. Л., Пашуева И. М. Статистико-геометрический анализ атомной структуры аморфных сплавов системы Re-Tb //Вестник Воронежского государственного технического университета. – 2017. – Т. 13. – №. 6.
	
	\bibitem{Zhang}
	Zhang Y. et al. Experimental and molecular dynamics simulation study of structure of liquid and amorphous Ni62Nb38 alloy //The Journal of chemical physics. – 2016. – Т. 145. – №. 20. – С. 204505.
	
	\bibitem{Mendelev}
	Wilson S. R., Mendelev M. I. Anisotropy of the solid–liquid interface properties of the Ni–Zr B33 phase from molecular dynamics simulation //Philosophical Magazine. – 2015. – Т. 95. – №. 2. – С. 224-241.
	

	
	

	
	\bibitem{tropin2011dependence}
	Tropin T. V., Schmelzer J. W. P., Schick C. On the dependence of the properties of glasses on cooling and heating rates: I. Entropy, entropy production, and glass transition temperature //Journal of Non-Crystalline Solids. – 2011. – Т. 357. – №. 4. – С. 1291-1302.
	
	\bibitem{aji2015kinetic}
	Aji D. P. B., Johari G. P. Kinetic-freezing and unfreezing of local-region fluctuations in a glass structure observed by heat capacity hysteresis //The Journal of chemical physics. – 2015. – Т. 142. – №. 21. – С. 214501.
	

	

	

	

	

	
\end{thebibliography}
%\printbibliography
%\bibliographystyle{utf8gost705u}
%\bibliography{lavrinenko,MBT18Lavrinenko,Elbrus17_LMV}

\end{document}
